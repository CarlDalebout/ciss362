Let $x,y$ be words in $\Sigma^*$ with $|x| = n + 1$.

We have the following:
\begin{align*}
           |x| &= n + 1                                                   & & \tag{a} \\ 
\THEREFORE |x| &\geq 1 \\
\THEREFORE |x| &\neq 0 \\
\THEREFORE x &\neq \answerbox{?}                                          & & \text{by Q2} \\
\THEREFORE x &= x'x'' \text{ for some } x' \in \Sigma, \ x'' \in \Sigma^* & & \text{by \answerbox{?}} \tag{b} \\
\THEREFORE |x| &= 1 + |x''| \text{ for some } x'' \in \Sigma^*            & & \text{by \answerbox{?}} \\
\THEREFORE n + 1 &= 1 + |x''| \text{ for some } x'' \in \Sigma^*          & & \text{by (a)} \\
\THEREFORE |x''| &= n \text{ for some } x'' \in \Sigma^*
\end{align*}

Therefore for some $x' \in \Sigma$ and $x'' \in \Sigma^*$
we have the following:
\begin{align*}
(xy)^R
&= \left( \answerbox{?}x'' y \right)^R                      & & \text{by (b)} \\
&= (\answerbox{?} \cdot x''y)^R                                               \\
&= (x''y)^R \cdot (\answerbox{?})^R                         & & \text{by $P(1)$}\\
&= \left( y^R \cdot (x'')^R \right) \cdot (\answerbox{?})^R & & \text{by $P(n)$} \\
&= y^R \cdot \left( (x'')^R \cdot (\answerbox{?})^R \right) & & \text{by \answerbox{?}}\\
&= y^R \cdot \left( \answerbox{?}x'' \right)^R              & & \text{by \answerbox{?}}\\
&= \answerbox{?}^R \cdot \answerbox{?}^R                    & & \text{by (b)} 
\end{align*}
Hence $P(n+1)$ holds.
\qed


\textsc{Note.}
The intuition behind the proof is that 
given $x$ of length $n + 1$,
we cut it up into $x'$ and $x''$ of lengths 1 and $n$.
We then using the inductive hypothesis $P(1)$ and $P(n)$.
Read over the proof again and make sure the see the strategy
in the proof.
Writing proofs formally is one thing.
Understanding the strategy in a proof is another.
Only by studying lots of proofs and understanding their strategy
then will you really understand how to construct convincing proofs
of your own.

We will prove that if $x \neq \ep$, then $|x| \neq 0$.
\begin{align*}
x &\neq \ep \\
\THEREFORE x &= x'x'' \text{ for some } x' \in \Sigma, x'' \in \Sigma^*
           & & \text{by \answerbox{?}} \\
\THEREFORE |x| &= |x' x''| \text{ for some } x' \in \Sigma, x'' \in \Sigma^* 
           \\ 
\THEREFORE |x| &= 1 + |x''| \text{ for some } x' \in \Sigma, x'' \in \Sigma^*
           & & \text{by \answerbox{?}} \\
\THEREFORE |x| &\geq 1 \text{ for some } x' \in \Sigma, x'' \in \Sigma^* 
\end{align*}
This implies that $|x| \geq 1$ and therefore $|x| \neq 0$.

Therefore we conclude that if $|x| = 0$, then $x = \ep$. \qed
\\

\textsc{Note.}
The less formal (but equally rigorous) way to write the proof is as follows:

\textit{
Suppose $x \neq \ep$, then by \answerbox{?}, $x = x'x''$ 
for some $x' \in \Sigma$ and $x'' \in \Sigma^*$.
In that case
\begin{align*}
|x| 
&= |x' \cdot x''| \\ 
&= 1 + |x''| & & \text{ by \answerbox{?}}    \\
&\geq 1
\end{align*}
which implies that $|x| \neq 0$.
}

\textsc{Note.}
Some basic facts are implicitly used.
\begin{enumerate}
\item If $x = y$ then $|x|= |y|$. 
\item If $n \geq 0$ then $n + 1 \geq 1$.
\item If $n \geq 1$ then $ n \neq 0$.
\item $\exists x(P)$ is the same as $P$ if
$P$ does not contain variables.
For instance \lq\lq there exists an $x$ such that
the sky is blue'' is the same as
\lq\lq the sky is blue''.
(Check existential generalization in your discrete math book.)
\end{enumerate}

\textsc{Note.}
Note that in this case instead of proving $P \implies Q$,
I prefer to prove $\lnot Q \implies \lnot P$.
Of course the two statements are the same, i.e.;
\[
(P \implies Q) \equiv (\lnot Q \implies \lnot P)
\]
How do you decide which one to prove: the left or the right?

To prove $P \implies Q$ directly means that I have to start with $P$.
I would then look at all the facts that begins with $P$.

If I want to prove $\lnot Q \implies \lnot P$, I would have to 
look at all the facts that I know that begins with $\lnot Q$.

I would compare the two above and see which one gives me more
facts to work with.
Sometimes it's obvious.
Sometimes it's not.
You might have a red herring where one gives you more facts, but
they end up (after a few steps of deduction) leading up to a deadend.
Sometimes either way work fine.

For the above case, it comes down to which of the following gives me
more tools to work with:
\begin{enumerate}
\item $|x| = 0$, or
\item $x \neq \ep$
\end{enumerate}


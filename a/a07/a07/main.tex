\input{myquizpreamble}
\input{yliow}
\input{ciss362}
\textwidth=6in

\renewcommand\TITLE{Assignment a05}




\renewcommand\AUTHOR{John Doe}

\begin{document}
\topmatter

\textsc{Objectives}
\begin{itemize}
\li Design DFAs.
\li Design NFAs.
\li Design regexes.
\li Show a language is regular by designing a DFA, NFA, or regex.
\li Show a language construction (operator) is a closed operator on
regular languages.
\end{itemize}

For questions asking you show an operation on regular
language is closed, you need not prove your construction is correct.
(But you are welcome to do so. Most of such proofs, if not immediate,
is by induction.)
Drawing a diagrma usually helps.
For solutions provided, make sure you study it carefully
as a guide for answering such questions.


\begin{myenum}
\li Sipser 1.31. Solution provided. Study it carefully. This is from a06. 
\li Sipser 1.32. Q1
\li Sipser 1.33. Q2
\li Sipser 1.34. Q3
\li Sipser 1.35. Q4
\li Sipser 1.36. Solution provided. Study it carefully.
\li Sipser 1.37. Q5
\li Sipser 1.38. Q6
\li Sipser 1.40. Q7. Solution to 1.40(a) is in the textbook. Study it carefully.
\li Sipser 1.41. Q8
\li Sipser 1.42. Q9
\end{myenum}

%------------------------------------------------------------------------------
\newpage
\textsc{How to draw a state diagram}

Here's an example showing you how to draw the elements of a state diagram.
Also, look at the solution to 1.3 below.

\begin{center}
  \begin{tikzpicture}[shorten >=1pt,>=triangle 60,double distance=2pt,node distance=2cm,auto,initial text=]
    
    \node[state,initial]   (q0) at (0, 0) {$q_0$};
    \node[state]           (q1) at (4, 0) {$q_1$};
    \node[state,accepting] (q2) at (8, 0) {$q_2$};
    \node[state]           (q3) at (0,-4) {$q_3$};
    \node[state]           (q4) at (4,-4) {$q_4$};

    \node[state]           (q5) at (12, 0) {$q_5$};
    \node[state]           (q6) at (12,-4) {$q_6$};

    \path[->]
    (q0) edge                node {$1$} (q1)    
    (q1) edge [bend left=15] node {$0$} (q2)
    (q0) edge [left]  node {$\alpha$} (q3)
    (q1) edge [right] node {$\beta$}  (q4)
    (q1) edge [loop above]   node {$1$} (q1)
    (q2) edge [loop below]   node {$0$} ()
    (q2) edge [bend left=15] node {$1$} (q1)
    (q3) edge [loop right]   node {$0,1$} ()
    (q4) edge [loop left]    node {$2,3$} ()
    (q5) edge [right, bend left=30]    node {$\gamma$} (q6)
    (q5) edge [right, bend left=60]    node {$\ep$}    (q6)
    (q5) edge [left, bend right=30]    node {$\delta$} (q6)
    ;
  \end{tikzpicture}
\end{center}

For more information on drawing state diagrams go to my tutorials and
look for \verb!latex-automata.pdf!:
\begin{center}
{\small \url{https://drive.google.com/file/d/1AeE-P0WNvQlitzPDxQpGE8bMR9Yc9gMW}}
\end{center}
Let me know if you have any questions about drawing state diagram.

%------------------------------------------------------------------------------
\newpage
Sipser 1.31. (This is from a06. Solution provided.) 

\textsc{Solution}.


We want to show that if $A$ is regular, then $A^R$ is also regular.

Let $M$ be a DFA for $A$:

\begin{center}
  \begin{tikzpicture}[shorten >=1pt,>=triangle 60,double distance=2pt,node distance=2cm,auto,initial text=]
    
    \node[state,initial]   (q0) at (0, 1) {$q_0$};
    
    \node[state]           (q1) at (2, 3) {$q_1$};
    \node[state]           (q2) at (2, 1) {$q_2$};
    \node[state]           (q3) at (2,-1) {$q_3$};

    \node[state]           (q5) at (4, 3) {$q_4$};
    \node[state]           (q6) at (4, 1) {$q_5$};
    \node[state]           (q7) at (4,-1) {$q_6$};

    \node[]                (q9) at (6, 3) {$...$};
    \node[]                (q10) at (6, 1) {$...$};
    \node[]                (q11) at (6, -1) {$...$};

    \node[state,accepting]           (q13) at (8, 3) {$q_7$};
    \node[state,accepting]           (q14) at (8, 1) {$q_8$};
    \node[state,accepting]           (q15) at (8, -1) {$q_9$};

    \node[state,white]   (s) at (10, 1) {};
    %\node[state,initial]   (s) at (10, 1) {$s$};

    \path[->]    
    (q0) edge                node {} (q1)
    (q0) edge                node {} (q2)
    (q0) edge                node {} (q3)
    
    (q1) edge                node {} (q5)
    (q2) edge                node {} (q6)
    (q3) edge                node {} (q7)
    
    (q5) edge                node {} (q9)
    (q6) edge                node {} (q10)
    (q7) edge                node {} (q11)

    (q9) edge                node {} (q13)
    (q10) edge               node {} (q14)
    (q11) edge               node {} (q15)
    ;
  \end{tikzpicture}
\end{center}
Suppose $M$ accepts the string $abaab$.
That means there's a path from the initial state of $M$ to an accept state that travels along a sequence of transitions
labeled $a, b, a, a, b$.
Suppose the accept state is $q_7$.
Then traveling in the reverse direction of this path, we will see the symbols $b,a,a,b,a$, going from $q_7$ to $q_0$.
Therefore we will need to construction an automata from $M$ where the direction of the transitions are reversed.
For this to be a valid automata, we can have only one start state.
$M$ might have multiple accept states.
We cannot simply choose anyone of the accept states of $M$.
That's not a problem: we will use nondeterminism to allow us to try all the accept states:

\begin{center}
  \begin{tikzpicture}[shorten >=1pt,>=triangle 60,double distance=2pt,node distance=2cm,auto,initial text=,initial where=right]
    
    \node[state,accepting]   (q0) at (0, 1) {$q_0$};
    
    \node[state]           (q1) at (2, 3) {$q_1$};
    \node[state]           (q2) at (2, 1) {$q_2$};
    \node[state]           (q3) at (2,-1) {$q_3$};

    \node[state]           (q5) at (4, 3) {$q_4$};
    \node[state]           (q6) at (4, 1) {$q_5$};
    \node[state]           (q7) at (4,-1) {$q_6$};

    \node[]                (q9) at (6, 3) {$...$};
    \node[]                (q10) at (6, 1) {$...$};
    \node[]                (q11) at (6, -1) {$...$};

    \node[state]           (q13) at (8, 3) {$q_7$};
    \node[state]           (q14) at (8, 1) {$q_8$};
    \node[state]           (q15) at (8, -1) {$q_9$};

    \node[state,initial]   (s) at (10, 1) {$s$};

    \path[->]    
    (q1) edge                node {} (q0)
    (q2) edge                node {} (q0)
    (q3) edge                node {} (q0)
    
    (q5) edge                node {} (q1)
    (q6) edge                node {} (q2)
    (q7) edge                node {} (q3)
    
    (q9) edge                node {} (q5)
    (q10) edge                node {} (q6)
    (q11) edge                node {} (q7)

    (q13) edge                node {} (q9)
    (q14) edge               node {} (q10)
    (q15) edge               node {} (q11)

    (s) edge                node[above right] {$\ep$} (q13)
    (s) edge               node[above] {$\ep$} (q14)
    (s) edge               node {$\ep$} (q15)
    ;
  \end{tikzpicture}
\end{center}

Note that in the new automata, the accept state is $q_0$.
That's the general idea.
We are now ready to construction our automata.

Let $M = (\Sigma, Q, q_0, F, \delta)$ be a DFA accepting $A$.
Define an NFA $N = (\Sigma, Q^R, s, F^R, \delta^R)$ where
$s$ is a new state (i.e., $s \not\in Q$),
$Q^R = Q \cup \{s\}$, $F^R = \{q_0\}$, and
\[
\delta^R: Q\cup\{s\} \times \Sigma_\ep \rightarrow P(Q\cup \{s\})
\]
is the transition function that behaves as stated above, i.e., they are
basically transitions from $M$ but with their directions reversed.
Furthermore there are new transitions from $s$ to all states in $F$.

First of all, in $N$, at state $s$, there are $\ep$--transitions to all the states in $F$.
Therefore
\[
\delta^R(s, \ep) = F
\]

Next, To describe the transitions of $N$ which are the reverse of transitions of $M$,
if in the DFA $M$, we have
\begin{center}
  \begin{tikzpicture}[shorten >=1pt,>=triangle 60,double distance=2pt,node distance=2cm,auto,initial text=,initial where=right]
    \node[state]           (q1) at (0, 0) {$q'$};
    \node[state]           (q2) at (2, 0) {$q$};
    \path[->]    
    (q1) edge                node {$c$} (q2)
    ;
  \end{tikzpicture}
\end{center}
the new automata $N$ will have
\begin{center}
  \begin{tikzpicture}[shorten >=1pt,>=triangle 60,double distance=2pt,node distance=2cm,auto,initial text=,initial where=right]
    \node[state]           (q1) at (0, 0) {$q'$};
    \node[state]           (q2) at (2, 0) {$q$};
    \path[->]    
    (q2) edge                node[above]  {$c$} (q1)
    ;
  \end{tikzpicture}
\end{center}
However, the definition of $\delta^R$ is not just
\begin{align*}
  \delta^R: Q \cup\{s\} \times \Sigma_\ep &\rightarrow P(Q\cup \{s\}) \\
  \delta^R(q, c)
  &=
  \begin{cases}
    \{q'\} & \text{ if $\delta(q', c) = q$} \\
    F      & \text{ if $q = s$ and $c = \ep$}
    \end{cases}
\end{align*}
for two reasons.
The first correction is due to the fact that $M$ we might have
\begin{center}
  \begin{tikzpicture}[shorten >=1pt,>=triangle 60,double distance=2pt,node distance=2cm,auto,initial text=,initial where=right]
    \node[state]           (q1) at (0, 1) {$q'$};
    \node[state]           (q3) at (0, -1) {$q''$};
    \node[state]           (q2) at (2, 0) {$q$};
    \path[->]    
    (q1) edge                node {$c$} (q2)
    (q3) edge                node {$c$} (q2)
    ;
  \end{tikzpicture}
\end{center}
In this case
\[
\delta^R(q, c) = \{q', q''\}
\]
More generally
\[
\delta^R(q, c) = \{ q' \in Q \mid \delta(q', c) = q \}
\]
Furthermore, this behavior of $\delta^R$ only applies to the case where $q \in Q$ and $c \in \Sigma$.
Therefore the transition function of the new automata should be modified to this:
\begin{align*}
  \delta^R: Q \cup\{s\} \times \Sigma_\ep &\rightarrow P(Q\cup \{s\}) \\
  \delta^R(q, c)
  &=
  \begin{cases}
    \{ q' \in Q \mid \delta(q', c) = q \} &  \text{ if $q \neq s$ and $c \neq \ep$} \\
    F      & \text{ if $q = s$ and $c = \ep$}
    \end{cases}
\end{align*}

The second correction is that $\delta^R$ is not complete since
it is not defined when $q = s$ and $c \neq \ep$ and when $q \neq s$ and $c = \ep$.
From the diagram above, you see that in the new automata there are no $\ep$--transitions other than
from state $s$.
In other word, we will need to fill in two blanks here:
\[
\delta^R(q, c)
=
\begin{cases}
  \{ q' \in Q \mid \delta(q', c) = q \} &  \text{ if $q \neq s$ and $c \neq \ep$} \\
  F      & \text{ if $q = s$ and $c = \ep$} \\
  ?      & \text{ if $q \neq s$ and $c = \ep$} \\
  ?      & \text{ if $q = s$ and $c \neq \ep$}
\end{cases}
\]
From our diagram above, you see that
$\delta^R(q, c) = \{\}$ for the last two cases.
The complete definition of $\delta^R$ is therefore
\begin{align*}
  \delta^R: Q \cup\{s\} \times \Sigma_\ep &\rightarrow P(Q\cup \{s\}) \\
  \delta^R(q, c)
  &=
  \begin{cases}
    \{ q' \in Q \mid \delta(q', c) = q \} & \text{if $q \neq s$ and $c \neq \ep$} \\ 
    F                                     & \text{if $q = s$ and $c = \ep$} \\
    \emptyset                             & \text{otherwise}
    \end{cases}
\end{align*}
\qed

\textsc{Notes}. Here are some DIYs.
\begin{enumerate}
\item Prove that if $w \in \Sigma^*$, then
  \[
  \delta^{R*}(s, w) - \{s\} = \{q \in Q \mid \delta^*(q, w^R) \in F\}
  \]
  (Hint: Induction on $|w|$.)
\item Prove formally that $L(N) = L(M)$.

\end{enumerate}


%------------------------------------------------------------------------------
\newpage
Q1. Sipser 1.32.

\textsc{Solution}.

(a.) %{w| w begins with a 1 and ends with a 0} \cup {w| w contains at least three 1s}
\begin{center}
  \begin{tikzpicture}[shorten >=1pt,>=triangle 60,double distance=2pt,node distance=2cm,auto,initial text=]

    \node[state,initial]   (q0)  at (0, 0)  {$q_0$};
    
    \node[state]           (q00)  at (2, 2)  {$q_00$};
    \node[state]           (q01)  at (4, 2)  {$q_01$};
    \node[state,accepting] (q02)  at (6, 2)  {$q_02$};
    
    \node[state]           (q10)  at (2, -2)  {$q_10$};
    \node[state]           (q11)  at (4, -2)  {$q_11$};
    \node[state]           (q12)  at (6, -2)  {$q_12$};
    \node[state,accepting] (q13)  at (8, -2)  {$q_13$};

    \path[->]
    (q0)  edge [bend left=30]  node {$\epsilon $} (q00)
    (q0)  edge [bend right=30] node {$\epsilon $} (q10)

    %--------------------------------------------------------    
    (q00) edge              node {$1$}      (q01)
   
    (q01) edge [bend left=15] node {$0$}    (q02)
    (q01) edge [loop above]   node {$1$}    ()
   
    (q02) edge [loop above]   node {$0$}    ()
    (q02) edge [bend left=15] node {$1$}    (q01)

    %--------------------------------------------------------    
    (q10) edge [loop above]   node {$0$}    ()
    (q10) edge                node {$1$}    (q11)
   
    (q11) edge [loop above]   node {$0$}    ()
    (q11) edge                node {$1$}    (q12)
   
    (q12) edge [loop above]   node {$0$}    ()
    (q12) edge                node {$1$}    (q13)

    (q13) edge [loop above]   node {$0,1$} ()
    ;
    
  \end{tikzpicture}
\end{center}

(b.) %{w| w contains the substring 0101 (i.e., w = x0101y for some x and y)} \cup {w| w doesn’t contain the substring 110}
\begin{center}
  \begin{tikzpicture}[shorten >=1pt,>=triangle 60,double distance=2pt,node distance=2cm,auto,initial text=]

    \node[state,initial]   (q0)  at (0, 0)  {$q_0$};
    
    \node[state]           (q00)  at (0, 2)  {$q_00$};
    \node[state]           (q01)  at (2, 2)  {$q_01$};
    \node[state,accepting] (q02)  at (4, 2)  {$q_02$};
    \node[state,accepting] (q03)  at (6, 2)  {$q_03$};
    \node[state,accepting] (q04)  at (8, 2)  {$q_04$};
    
    \node[state,accepting] (q10)  at (0, -2)  {$q_10$};
    \node[state,accepting] (q11)  at (2, -2)  {$q_11$};
    \node[state,accepting] (q12)  at (4, -2)  {$q_12$};
    \node[state]           (q13)  at (6, -2)  {$q_13$};

    \path[->]
    (q0)  edge [left] node {$\epsilon $} (q00)
    (q0)  edge [left] node {$\epsilon $} (q10)

    %--------------------------------------------------------    
    (q00) edge                 node {$0$}   (q01)
    (q00) edge [loop above]    node {$1$}   ()
    
    (q01) edge                 node {$0$}   (q02)
    (q01) edge [loop above]    node {$0$}   ()
    
    (q02) edge                 node {$0$}   (q03)
    (q02) edge [bend left=45]  node {$1$}   (q00)
    
    (q03) edge [bend left=45]  node {$0$}   (q01)
    (q03) edge                 node {$1$}   (q04)
    
    (q04) edge [loop above]    node {$0,1$} ()
    %--------------------------------------------------------    
    (q10) edge [loop below]    node {$0$}   ()
    (q10) edge                 node {$1$}   (q11)
    
    (q11) edge [bend left=30]  node {$0$}   (q10)
    (q11) edge                 node {$1$}   (q12)
    
    (q12) edge                 node {$0$}   (q13)
    (q12) edge [loop above]    node {$1$}   ()

    (q13) edge [loop above]    node {$0,1$} ()
    ;

  \end{tikzpicture}
\end{center}

%------------------------------------------------------------------------------
\newpage
Q2. Sipser 1.33.

\textsc{Solution}.

(a.) %{w| the length of w is at most 5} \dot {w| every odd position of w is a 1}
\begin{center}
  \begin{tikzpicture}[shorten >=1pt,>=triangle 60,double distance=2pt,node distance=2cm,auto,initial text=]

    \node[state,initial]   (q00)  at (0, 0)  {$q_00$};
    \node[state]           (q01)  at (2, 0)  {$q_01$};
    \node[state]           (q02)  at (4, 0)  {$q_02$};
    \node[state]           (q03)  at (6, 0)  {$q_03$};
    \node[state]           (q04)  at (8, 0)  {$q_04$};
    \node[state]           (q05)  at (10, 0) {$q_05$};
    \node[state]           (q06)  at (12, 0) {$q_06$};
    

    \node[state,accepting] (q10)  at (0, -4) {$q_10$};
    \node[state,accepting] (q11)  at (2, -4) {$q_11$};
    \node[state,accepting] (q12)  at (4, -4) {$q_12$};
    \node[state]           (q13)  at (6, -4) {$q_13$};

    \path[->]
    (q00) edge                 node {$0,1$}      (q01)
    (q00) edge [left]          node {$\epsilon$} (q10)
    
    (q01) edge                 node {$0,1$}      (q02)
    (q01) edge [bend right=15] node {$\epsilon$} (q10)
   
    (q02) edge                 node {$0,1$}      (q03)
    (q02) edge [bend right=15] node {$\epsilon$} (q10)
   
    (q03) edge                 node {$0,1$}      (q04)
    (q03) edge [bend right=15] node {$\epsilon$} (q10)
   
    (q04) edge                 node {$0,1$}      (q05)
    (q04) edge [bend right=15] node {$\epsilon$} (q10)
   
    (q05) edge                 node {$0,1$}      (q06)
    (q05) edge [bend right=15] node {$\epsilon$} (q10)

    (q06) edge [loop above]    node {$0,1$}      ()

    %-----------------------------------------------------
    (q10) edge                 node {$1$}        (q11)
    
    (q11) edge [bend left=15]  node {$0,1$}      (q12)
    
    (q12) edge                 node {$0$}        (q13)
    (q12) edge [bend left=15]  node {$1$}        (q11)
    
    (q13) edge [loop above]    node {$0,1$}      (q11)
    ;
    
  \end{tikzpicture}
\end{center}


(b.) %{w| w contains at least three 1s} \dot {}
\begin{center}
  \begin{tikzpicture}[shorten >=1pt,>=triangle 60,double distance=2pt,node distance=2cm,auto,initial text=]
    
    \node[state,initial]   (q0)  at (0, 0)  {$q_0$};
    \node[state]           (q1)  at (2, 0)  {$q_1$};
    \node[state]           (q2)  at (4, 0)  {$q_2$};
    \node[state]           (q3)  at (6, 0)  {$q_3$};
    \node[state,accepting] (q4)  at (8, 0)  {$q_4$};
    
    \path[->]
    (q0) edge [loop above]    node {$0$}        ()
    (q0) edge                 node {$1$}        (q1)

    (q1) edge [loop above]    node {$0$}        ()
    (q1) edge                 node {$1$}        (q2)

    (q2) edge [loop above]    node {$0$}        ()
    (q2) edge                 node {$1$}        (q3)

    (q3) edge [loop above]    node {$0$}        ()
    (q3) edge                 node {$\epsilon$} (q4)
    ;

  \end{tikzpicture}
\end{center}

%------------------------------------------------------------------------------
\newpage
Q3. Sipser 1.34.

\textsc{Solution}.


(b)
\begin{center}
  \begin{tikzpicture}[shorten >=1pt,>=triangle 60,double distance=2pt,node distance=2cm,auto,initial text=]
    
    \node[state,initial]   (q0) at (0, 0) {$q_0$};
    \node[state]           (q1) at (2, 0) {$q_0$};
    \node[state]           (q2) at (4, 0) {$q_0$};
    \node[state]           (q3) at (6, 0) {$q_0$};
    \node[state,accepting] (q4) at (8, 0) {$q_0$};

    \path[->]
    (q0) edge                 node {$0$}   (q1)
    (q0) edge [loop above]    node {$1$}   ()
    
    (q1) edge [bend left=45]  node {$0$}   (q0)
    (q1) edge                 node {$1$}   (q2)
    
    (q2) edge                 node {$0$}   (q3)
    (q2) edge [bend right=45] node {$1$}   (q0)
    
    (q3) edge [bend left=45]  node {$0$}   (q0)
    (q3) edge                 node {$1$}   (q4)
    
    (q4) edge [loop right]    node {$1,0$} ()
    ;
  \end{tikzpicture}
\end{center}

(c)
\begin{center}
  \begin{tikzpicture}[shorten >=1pt,>=triangle 60,double distance=2pt,node distance=2cm,auto,initial text=]
    
    \node[state,initial,accepting] (q0) at (0, 0)  {$q_0$};
    \node[state]                   (q1) at (0, -4) {$q_1$};
    \node[state,accepting]         (q2) at (3, 0)  {$q_2$};
    \node[state]                   (q3) at (3, -4) {$q_3$};
    \node[state,accepting]         (q4) at (6, 0)  {$q_4$};
    \node[state,accepting]         (q5) at (6, -4) {$q_5$};
    
    \path[->]
    (q0) edge [bend left=15]          node {$0$}   (q1)
    (q0) edge                         node {$1$}   (q2)

    (q1) edge [bend left=15]          node {$0$}   (q0)
    (q1) edge                         node {$1$}   (q3)

    (q2) edge [bend left=15]          node {$0$}   (q3)
    (q2) edge                         node {$1$}   (q4)

    (q3) edge [bend left=15]          node {$0$}   (q2)
    (q3) edge                         node {$1$}   (q5)
    
    (q4) edge [bend left=15]          node {$0$}   (q5)
    
    (q5) edge [bend left=15]          node {$0$}   (q4)
    ;
  \end{tikzpicture}
\end{center}

(d)
\begin{center}
  \begin{tikzpicture}[shorten >=1pt,>=triangle 60,double distance=2pt,node distance=2cm,auto,initial text=]
    
    \node[state,initial]   (q0) at (0, 0) {$q_0$};
    \node[state,accepting] (q1) at (4, 0) {$q_1$};

    \path[->]
    (q0) edge node {$0$}   (q1)
    ;
  \end{tikzpicture}
\end{center}



(e)
\begin{center}
  \begin{tikzpicture}[shorten >=1pt,>=triangle 60,double distance=2pt,node distance=2cm,auto,initial text=]
    
    \node[state,initial]   (q0) at (0, 0) {$q_0$};
    \node[state,accepting] (q1) at (3, 0) {$q_1$};
    \node[state]           (q2) at (6, 0) {$q_2$};

    \path[->]
    (q0) edge                 node {$0$} (q1)
    (q0) edge [bend right=30] node {$1$} (q2)
    
    (q1) edge [loop above]    node {$0$} ()
    (q1) edge [bend left=15]  node {$1$} (q2)
    
    (q2) edge [bend left=15]  node {$0$} (q1)
    (q2) edge [loop right]    node {$1$} ()

    ;
  \end{tikzpicture}
\end{center}

\newpage
(g)
\begin{center}
  \begin{tikzpicture}[shorten >=1pt,>=triangle 60,double distance=2pt,node distance=2cm,auto,initial text=]
    
    \node[state,initial,accepting] (q0) at (0, 0) {$q_0$};

    \path[->]
    ;
  \end{tikzpicture}
\end{center}

(h)
\begin{center}
  \begin{tikzpicture}[shorten >=1pt,>=triangle 60,double distance=2pt,node distance=2cm,auto,initial text=]
    
    \node[state,initial,accepting] (q0) at (0, 0) {$q_0$};

    \path[->]
    (q0) edge [loop right] node {$0$} ()
    ;
  \end{tikzpicture}
\end{center}


%------------------------------------------------------------------------------
\newpage
Q4. Sipser 1.35.

\textsc{Solution}.

Regular Expression: $D = \{\beta(\beta\beta)^*(\alpha\alpha)^*\}$

\begin{center}
  \begin{tikzpicture}[shorten >=1pt,>=triangle 60,double distance=2pt,node distance=2cm,auto,initial text=]
    
    \node[state,initial]   (q0) at (0,0) {$q_0$};
    \node[state,accepting] (q1) at (2,0) {$q_1$};
    \node[state]           (q2) at (4,0) {$q_2$};
    \node[state,accepting] (q3) at (6,0) {$q_3$};
    \node[state]           (q4) at (8,0) {$q_4$};
    
    \path[->]
    (q0) edge [bend right=45] node {$\alpha$}       (q4)
    (q0) edge [bend left=15]  node {$\beta$}        (q1)
    
    (q1) edge                 node {$\alpha$}       (q2)
    (q1) edge [bend left=15]  node {$\beta$}        (q0)
    
    (q2) edge [bend left=15]  node {$\alpha$}       (q3)
    (q2) edge [bend right=45] node {$\beta$}        (q4)
    
    (q3) edge [bend left=15]  node {$\alpha$}       (q2)
    (q3) edge                 node {$\beta$}        (q4)
    
    (q4) edge [loop right]    node {$\alpha,\beta$} ()
    ;
    
  \end{tikzpicture}
\end{center}

%------------------------------------------------------------------------------
\newpage
Sipser 1.36. Solution provided. Study it carefully.

(Hint: In this case, it's easier not to use automata construction.
Use a regular expression instead.)

\textsc{Solution}.
 
\textsc{Solution 1:}
Let $n \geq 1$.
We want to show 
\[
B_n = \{a^k \,|\, k \text{ is a multiple of $n$} \}
\] 
is regular. 
Define the following DFA $M$ (over $\Sigma = \{a\}$) as follows:
\begin{center}
\begin{tikzpicture}[shorten >=1pt,node distance=2cm,auto,initial text=]
\node[state,initial,accepting] (A) at (  0,  0) {$q_0$};
\node[state] (B) at (2, 0) {$q_1$};
\node[state] (C) at (4, 0) {$q_2$};
\draw(6, 0) node (D) {$...$};
\node[state] (E) at (8, 0) {$q_{n-1}$};

\path[->]
(A) edge [] node {$a$} (B)
(B) edge [] node {$a$} (C)
(C) edge [] node {$a$} (D)
(D) edge [] node {$a$} (E)
(E) edge [bend right=40,above] node {$a$} (A)
;
\end{tikzpicture}
\end{center}
Formally the DFA $M$ is defined as
$M = (\Sigma, Q, q_0, \delta, F)$ where:
\begin{enumerate}[topsep=0in,parsep=0in]
\item $\Sigma = \{a\}$
\item $Q = \{q_0, q_1, \ldots, q_{n-1}\}$
\item $\delta : Q \times \Sigma \rightarrow Q$ is defined by 
 \[
 \delta(q_i, a)
 =
 \begin{cases}
 q_{i + 1} & \text{if } 0 \leq i < n - 1 \\
 q_0       & \text{if } i = n - 1
 \end{cases}
 \]
\item $F = \{q_0\}$
\end{enumerate}
Clearly the strings accepted by $M$ are $\epsilon$, $a^n$, $a^{2n}$, $\ldots$,
i.e. $L(M) = \{a^k \,|\, \text{$k$ is a multiple of $n$}\} = B_n$.
Hence $B_n$ must be regular.

\textsc{Solution 2:}
Let $n \geq 1$.
We want to show 
\[
B_n = \{a^k \,|\, k \text{ is a multiple of $n$} \}
\] 
is regular. Note that 

\begin{align*}
B_n 
&= \{a^k \,|\, k \text{ is a multiple of $n$} \} \\
&= \{a^{mn} \,|\, m \geq 0 \} \\
&= \{(a^n)^m \,|\, m \geq 0 \} \\
&= \{a^n\}^* \\
&= L(a^n)^* \\
&= L((a^n)^*) \\
\end{align*}
i.e. $B_n$ is the language accepted by the regular expression $r = (a^n)^*$.
We already know that the language generated by a regular expression is also
accepted by a DFA.
Hence $B_n$ must be regular.

{\textsc Note.}
Note that the second solution is a lot clearer since
$L(r) = \{a^k \mid k \text{ is a multiple of $n$} \}$ is
shown completely.
However in the first solution the statement
\[
L(M) = \{a^m \mid m \text{ is a multiple of $n$} \}
\]
is not so immediate and properly speaking requires some proof.
You can formally show that 
\[
L(M) = \{a^m \mid m \text{ is a multiple of $n$} \}
\]
using mathematical induction.
This is the reason why in CS and Math, we frequently have several different
ways of looking at the same concept.
(In the case of regular languages, a regular language is one that
is accepted by or DFA, \textit{or} is accepted by an NFA, \textit{or}
is generated by a regular expression.)
Sometimes one way of looking at a problem will yield a more natural 
solution/proof or one that is shorter.


%------------------------------------------------------------------------------
\newpage
Q5. Sipser 1.37.

\textsc{Solution}.

\begin{center}
  \begin{tikzpicture}[shorten >=1pt,>=triangle 60,double distance=2pt,node distance=2cm,auto,initial text=]

    \node[state,initial,accepting] (q0)  at (0, 0)  {$q_0$};
    \node[state,accepting]         (q1)  at (2, 0)  {$q_1$};
    \node[state,accepting]         (q2)  at (4, 0)  {$q_2$};
    \node[state,accepting]         (q3)  at (2, -2) {$q_3$};
    \node[state]                   (q4)  at (4, -2) {$q_4$};

    \path[->]
    (q0) edge [loop above]    node {$0$}   ()
    (q0) edge                 node {$1$}   (q1)
    
    (q1) edge [bend left=15]  node {$0$}   (q2)
    (q1) edge [left]          node {$1$}   (q3)
    
    (q2) edge [bend left=15]  node {$0$}   (q1)
    (q2) edge [left]          node {$1$}   (q4)
    
    (q3) edge [loop below]    node {$0$}   ()
    (q3) edge                 node {$1$}   (q4)
    
    (q4) edge [loop below]    node {$0,1$} ()
    ;
    
  \end{tikzpicture}
\end{center}


%------------------------------------------------------------------------------
\newpage
Q6. Sipser 1.38.

\textsc{Solution}.

(a)
Let $M = (\Sigma, Q, q_0, F, \delta)$.
Define $\overline{M}$ to be the DFA
\[
\overline{M} = (\Sigma, Q, q_0, Q - F, \delta)
\]
We will show that
\[
L(\overline{M}) = \overline{L(M)}
\]
Let $w \in \Sigma^*$.
\begin{align*}
  w \in L(\overline{M})
  &\iff ? \\
  &\iff ? \\
  &\iff w \in \overline{L(M)} \\
\end{align*}

(Recall that $w \in L(M)$ iff $\delta^*(q_0, w) \in F$.)

(b)


%------------------------------------------------------------------------------
\newpage
Q7. Sipser 1.40. Solution to 1.40(a) is in the textbook -- study it carefully.

\textsc{Solution}.


(a)

(b)

(c)



%------------------------------------------------------------------------------
\newpage
Q8. Sipser 1.41.

\textsc{Solution}.

(a) $T$ derives strings containing any number of $0$s (including none) and
exactly one $\#$.
Therefore $TT$ derives strings with any number of $0$s and exactly two
$\#$s.
$U$ derives strings of the form $0^n \# 0^{2n}$.
Hence $L(G)$ contains strings with any number of $0$s and exactly two
$\#$ or strings of the form $0^n \# 0^{2n}$, i.e.,
\[
L(G) = L(0^* \# 0^* \# 0^*) \cup \{0^n \# 0^{2n} \mid n \geq 0\} 
\]


%------------------------------------------------------------------------------
\newpage
Q9. Sipser 1.42.

\textsc{Solution}.

\input{q09.tex}

\end{document}


We want to show that if $A$ is regular, then $A^R$ is also regular.

Let $M$ be a DFA for $A$:

\begin{center}
  \begin{tikzpicture}[shorten >=1pt,>=triangle 60,double distance=2pt,node distance=2cm,auto,initial text=]
    
    \node[state,initial]   (q0) at (0, 1) {$q_0$};
    
    \node[state]           (q1) at (2, 3) {$q_1$};
    \node[state]           (q2) at (2, 1) {$q_2$};
    \node[state]           (q3) at (2,-1) {$q_3$};

    \node[state]           (q5) at (4, 3) {$q_4$};
    \node[state]           (q6) at (4, 1) {$q_5$};
    \node[state]           (q7) at (4,-1) {$q_6$};

    \node[]                (q9) at (6, 3) {$...$};
    \node[]                (q10) at (6, 1) {$...$};
    \node[]                (q11) at (6, -1) {$...$};

    \node[state,accepting]           (q13) at (8, 3) {$q_7$};
    \node[state,accepting]           (q14) at (8, 1) {$q_8$};
    \node[state,accepting]           (q15) at (8, -1) {$q_9$};

    \node[state,white]   (s) at (10, 1) {};
    %\node[state,initial]   (s) at (10, 1) {$s$};

    \path[->]    
    (q0) edge                node {} (q1)
    (q0) edge                node {} (q2)
    (q0) edge                node {} (q3)
    
    (q1) edge                node {} (q5)
    (q2) edge                node {} (q6)
    (q3) edge                node {} (q7)
    
    (q5) edge                node {} (q9)
    (q6) edge                node {} (q10)
    (q7) edge                node {} (q11)

    (q9) edge                node {} (q13)
    (q10) edge               node {} (q14)
    (q11) edge               node {} (q15)
    ;
  \end{tikzpicture}
\end{center}
Suppose $M$ accepts the string $abaab$.
That means there's a path from the initial state of $M$ to an accept state that travels along a sequence of transitions
labeled $a, b, a, a, b$.
Suppose the accept state is $q_7$.
Then traveling in the reverse direction of this path, we will see the symbols $b,a,a,b,a$, going from $q_7$ to $q_0$.
Therefore we will need to construction an automata from $M$ where the direction of the transitions are reversed.
For this to be a valid automata, we can have only one start state.
$M$ might have multiple accept states.
We cannot simply choose anyone of the accept states of $M$.
That's not a problem: we will use nondeterminism to allow us to try all the accept states:

\begin{center}
  \begin{tikzpicture}[shorten >=1pt,>=triangle 60,double distance=2pt,node distance=2cm,auto,initial text=,initial where=right]
    
    \node[state,accepting]   (q0) at (0, 1) {$q_0$};
    
    \node[state]           (q1) at (2, 3) {$q_1$};
    \node[state]           (q2) at (2, 1) {$q_2$};
    \node[state]           (q3) at (2,-1) {$q_3$};

    \node[state]           (q5) at (4, 3) {$q_4$};
    \node[state]           (q6) at (4, 1) {$q_5$};
    \node[state]           (q7) at (4,-1) {$q_6$};

    \node[]                (q9) at (6, 3) {$...$};
    \node[]                (q10) at (6, 1) {$...$};
    \node[]                (q11) at (6, -1) {$...$};

    \node[state]           (q13) at (8, 3) {$q_7$};
    \node[state]           (q14) at (8, 1) {$q_8$};
    \node[state]           (q15) at (8, -1) {$q_9$};

    \node[state,initial]   (s) at (10, 1) {$s$};

    \path[->]    
    (q1) edge                node {} (q0)
    (q2) edge                node {} (q0)
    (q3) edge                node {} (q0)
    
    (q5) edge                node {} (q1)
    (q6) edge                node {} (q2)
    (q7) edge                node {} (q3)
    
    (q9) edge                node {} (q5)
    (q10) edge                node {} (q6)
    (q11) edge                node {} (q7)

    (q13) edge                node {} (q9)
    (q14) edge               node {} (q10)
    (q15) edge               node {} (q11)

    (s) edge                node[above right] {$\ep$} (q13)
    (s) edge               node[above] {$\ep$} (q14)
    (s) edge               node {$\ep$} (q15)
    ;
  \end{tikzpicture}
\end{center}

Note that in the new automata, the accept state is $q_0$.
That's the general idea.
We are now ready to construction our automata.

Let $M = (\Sigma, Q, q_0, F, \delta)$ be a DFA accepting $A$.
Define an NFA $N = (\Sigma, Q^R, s, F^R, \delta^R)$ where
$s$ is a new state (i.e., $s \not\in Q$),
$Q^R = Q \cup \{s\}$, $F^R = \{q_0\}$, and
\[
\delta^R: Q\cup\{s\} \times \Sigma_\ep \rightarrow P(Q\cup \{s\})
\]
is the transition function that behaves as stated above, i.e., they are
basically transitions from $M$ but with their directions reversed.
Furthermore there are new transitions from $s$ to all states in $F$.

First of all, in $N$, at state $s$, there are $\ep$--transitions to all the states in $F$.
Therefore
\[
\delta^R(s, \ep) = F
\]

Next, To describe the transitions of $N$ which are the reverse of transitions of $M$,
if in the DFA $M$, we have
\begin{center}
  \begin{tikzpicture}[shorten >=1pt,>=triangle 60,double distance=2pt,node distance=2cm,auto,initial text=,initial where=right]
    \node[state]           (q1) at (0, 0) {$q'$};
    \node[state]           (q2) at (2, 0) {$q$};
    \path[->]    
    (q1) edge                node {$c$} (q2)
    ;
  \end{tikzpicture}
\end{center}
the new automata $N$ will have
\begin{center}
  \begin{tikzpicture}[shorten >=1pt,>=triangle 60,double distance=2pt,node distance=2cm,auto,initial text=,initial where=right]
    \node[state]           (q1) at (0, 0) {$q'$};
    \node[state]           (q2) at (2, 0) {$q$};
    \path[->]    
    (q2) edge                node[above]  {$c$} (q1)
    ;
  \end{tikzpicture}
\end{center}
However, the definition of $\delta^R$ is not just
\begin{align*}
  \delta^R: Q \cup\{s\} \times \Sigma_\ep &\rightarrow P(Q\cup \{s\}) \\
  \delta^R(q, c)
  &=
  \begin{cases}
    \{q'\} & \text{ if $\delta(q', c) = q$} \\
    F      & \text{ if $q = s$ and $c = \ep$}
    \end{cases}
\end{align*}
for two reasons.
The first correction is due to the fact that $M$ we might have
\begin{center}
  \begin{tikzpicture}[shorten >=1pt,>=triangle 60,double distance=2pt,node distance=2cm,auto,initial text=,initial where=right]
    \node[state]           (q1) at (0, 1) {$q'$};
    \node[state]           (q3) at (0, -1) {$q''$};
    \node[state]           (q2) at (2, 0) {$q$};
    \path[->]    
    (q1) edge                node {$c$} (q2)
    (q3) edge                node {$c$} (q2)
    ;
  \end{tikzpicture}
\end{center}
In this case
\[
\delta^R(q, c) = \{q', q''\}
\]
More generally
\[
\delta^R(q, c) = \{ q' \in Q \mid \delta(q', c) = q \}
\]
Furthermore, this behavior of $\delta^R$ only applies to the case where $q \in Q$ and $c \in \Sigma$.
Therefore the transition function of the new automata should be modified to this:
\begin{align*}
  \delta^R: Q \cup\{s\} \times \Sigma_\ep &\rightarrow P(Q\cup \{s\}) \\
  \delta^R(q, c)
  &=
  \begin{cases}
    \{ q' \in Q \mid \delta(q', c) = q \} &  \text{ if $q \neq s$ and $c \neq \ep$} \\
    F      & \text{ if $q = s$ and $c = \ep$}
    \end{cases}
\end{align*}

The second correction is that $\delta^R$ is not complete since
it is not defined when $q = s$ and $c \neq \ep$ and when $q \neq s$ and $c = \ep$.
From the diagram above, you see that in the new automata there are no $\ep$--transitions other than
from state $s$.
In other word, we will need to fill in two blanks here:
\[
\delta^R(q, c)
=
\begin{cases}
  \{ q' \in Q \mid \delta(q', c) = q \} &  \text{ if $q \neq s$ and $c \neq \ep$} \\
  F      & \text{ if $q = s$ and $c = \ep$} \\
  ?      & \text{ if $q \neq s$ and $c = \ep$} \\
  ?      & \text{ if $q = s$ and $c \neq \ep$}
\end{cases}
\]
From our diagram above, you see that
$\delta^R(q, c) = \{\}$ for the last two cases.
The complete definition of $\delta^R$ is therefore
\begin{align*}
  \delta^R: Q \cup\{s\} \times \Sigma_\ep &\rightarrow P(Q\cup \{s\}) \\
  \delta^R(q, c)
  &=
  \begin{cases}
    \{ q' \in Q \mid \delta(q', c) = q \} & \text{if $q \neq s$ and $c \neq \ep$} \\ 
    F                                     & \text{if $q = s$ and $c = \ep$} \\
    \emptyset                             & \text{otherwise}
    \end{cases}
\end{align*}
\qed

\textsc{Notes}. Here are some DIYs.
\begin{enumerate}
\item Prove that if $w \in \Sigma^*$, then
  \[
  \delta^{R*}(s, w) - \{s\} = \{q \in Q \mid \delta^*(q, w^R) \in F\}
  \]
  (Hint: Induction on $|w|$.)
\item Prove formally that $L(N) = L(M)$.

\end{enumerate}

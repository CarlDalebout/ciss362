%\makeatletter
%\DeclareOldFontCommand{\rm}{\normalfont\rmfamily}{\mathrm}
%\DeclareOldFontCommand{\sf}{\normalfont\sffamily}{\mathsf}
%\DeclareOldFontCommand{\tt}{\normalfont\ttfamily}{\mathtt}
%\DeclareOldFontCommand{\bf}{\normalfont\bfseries}{\mathbf}
%\DeclareOldFontCommand{\it}{\normalfont\itshape}{\mathit}
%\DeclareOldFontCommand{\sl}{\normalfont\slshape}{\@nomath\sl}
%\DeclareOldFontCommand{\sc}{\normalfont\scshape}{\@nomath\sc}
%\makeatother

\input{myassignmentpreamble}
\input{yliow}
\input{ciss362}
\textwidth=6in

\renewcommand\TITLE{Assignment a03}
\newcommand\tf{T or F or M}
\newcommand\answerbox[1]{\textbox{#1}}
\newcommand\codebox[1]{\begin{console}#1\end{console}}

\usepackage{pifont}
\newcommand{\cmark}{\textred{\ding{51}}}
\newcommand{\xmark}{\textred{\ding{55}}}

\newcounter{qc}
\newcommand\nextq{
%\newpage
\addtocounter{qc}{1}
Q{\theqc}.
}

\DefineVerbatimEnvironment%
 {answercode}{Verbatim}
 {frame=single}

\newenvironment{largebox}[1]{%
 \boxparone{#1}
}
{}




\usepackage{environ}
\let\oldquote=\quote
\let\endoldquote=\endquote
\let\quote\relax
\let\endquote\relax

\NewEnviron{answerlong}%
  {\global\let\tmp\BODY\aftergroup\doboxpar}

\newcommand\doboxpar{%
  \let\quote=\oldquote
  \let\endquote=\endoldquote
  \boxpar{\tmp}
}







\newenvironment{mcq}[7]%
{% begin code
#1 \dotfill{#2}
 \begin{tightlist}
 \item[(A)] #3
 \item[(B)] #4
 \item[(C)] #5
 \item[(D)] #6
 \item[(E)] #7 
 \end{tightlist}
}%
{% end code
} 


\renewcommand\AUTHOR{John Doe}

\begin{document}
\topmatter

\textsc{Objectives}
\begin{itemize}
\li Design context-free grammars.
\li Design PDA.
\end{itemize}


For questions asking you show an operation on context-free languages
language is closed, you need not prove your construction is correct.
(But you are welcome to do so. Most of such proofs, if not immediate,
is by induction.)
If the question does not say so, you can choose to give a PDA or a
context-free grammar.
For a PDA, draw the PDA state diagram and give the formal definition.

Study the solutions to the following questions in the textbook:
2.3, 2.4 (a) and (d), 2.6(a) and (c), 2.7, 2.8.

Here are the questions you should work.
Some questions have solutions (either from
the book or I have written up the solution).
\begin{myenum}
\li Sipser 2.1. Q1. Solution to 2.1(b) is provided. Study it carefully. 
\li Sipser 2.2. Q2. You may assume that $\{a^nb^nc^n \mid n \geq 0\}$ is not a
context-free language.
\li Sipser 2.3. Solution is provided in the textbook. Study it carefully.
\li Sipser 2.4. Q3. Solutions to 2.4(a) and 2.4(d) are provided in the textbook.
Study it carefully.
\li Sipser 2.5. Q4.
\li Sipser 2.6. Q5. Solutions to 2.6(a) and 2.6(c) are provided in the textbook. Study it carefully.
\li Sipser 2.7. Solution to 2.7 is provided in the textbook. Study it carefully.
\li Sipser 2.8. Solution to 2.8 is provided in the textbook. Study it carefully.
\li Sipser 2.9. Q6.
\li Sipser 2.10. Q7.
\li Sipser 2.13. Q8. Solution to 2.13(a) is provided. 
\end{myenum}

%------------------------------------------------------------------------------
\newpage
\textsc{How to draw a state diagram}

Here's an example showing you how to draw the elements of a state diagram.
Also, look at the solution to 1.3 below.

\begin{center}
  \begin{tikzpicture}[shorten >=1pt,>=triangle 60,double distance=2pt,node distance=2cm,auto,initial text=]
    
    \node[state,initial]   (q0) at (0, 0) {$q_0$};
    \node[state]           (q1) at (4, 0) {$q_1$};
    \node[state,accepting] (q2) at (8, 0) {$q_2$};
    \node[state]           (q3) at (0,-4) {$q_3$};
    \node[state]           (q4) at (4,-4) {$q_4$};

    \node[state]           (q5) at (12, 0) {$q_5$};
    \node[state]           (q6) at (12,-4) {$q_6$};

    \path[->]
    (q0) edge                node {$1$} (q1)    
    (q1) edge [bend left=15] node {$0$} (q2)
    (q0) edge [left]  node {$\alpha$} (q3)
    (q1) edge [right] node {$\beta$}  (q4)
    (q1) edge [loop above]   node {$1$} (q1)
    (q2) edge [loop below]   node {$0$} ()
    (q2) edge [bend left=15] node {$1$} (q1)
    (q3) edge [loop right]   node {$0,1$} ()
    (q4) edge [loop left]    node {$2,3$} ()
    (q5) edge [right, bend left=30]    node {$\gamma$} (q6)
    (q5) edge [right, bend left=60]    node {$\ep$}    (q6)
    (q5) edge [left, bend right=30]    node {$\delta$} (q6)
    ;
  \end{tikzpicture}
\end{center}

For more information on drawing state diagrams go to my tutorials at
\url{http://yliow.github.io} and 
look for \verb!latex-automata.pdf!:
Let me know if you have any questions about drawing state diagram.


%------------------------------------------------------------------------------
\newpage
\textsc{How to write a context-free grammar}

Here's a context-free grammar, $G$, for our $\{ a^n b^n \mid n \geq 0 \}$:
\[
G:
\begin{cases}
S \rightarrow \ep \\
S \rightarrow aSb
\end{cases}
\]
Usually we combine the productions for the same variable like this:
\[
G:
S \rightarrow \ep \mid aSb
\]
Here's an example of a longer grammar:
\[
G_1:
\begin{cases}
S \rightarrow \ep \mid TU \\
T \rightarrow a \mid   aT \\
U \rightarrow \ep \mid bUcc
\end{cases}
\]
And remember to list the starting variable first.

%------------------------------------------------------------------------------
\newpage
Q1. Sipser 2.1. 

\textsc{Solution}.

(a.) %{01, 001, 010}^{*}
\begin{center}
  \begin{tikzpicture}[shorten >=1pt,>=triangle 60,double distance=2pt,node distance=2cm,auto,initial text=]

    \node[state,initial,accepting] (q0)  at (0, 0)  {$q_0$};
    \node[state]                   (q1)  at (2, 0)  {$q_1$};
    \node[state]                   (q2)  at (4, 0)  {$q_2$};
    \node[state,accepting]         (q3)  at (6, 0)  {$q_3$};
    \node[state,accepting]         (q4)  at (2, 2)  {$q_4$};
    \node[state,accepting]         (q5)  at (2, 4)  {$q_5$};

    \path[->]
    (q0) edge                 node {$0$}        (q1)
    
    (q1) edge                 node {$0$}        (q2)
    (q1) edge [left]          node {$1$}        (q4)
    
    (q2) edge                 node {$1$}        (q3)
    
    (q3) edge [bend left=45]  node {$\epsilon$} (q0)
    
    (q4) edge [left]          node {$0$}        (q5)
    (q4) edge [bend right=45] node {$\epsilon$} (q0)
    
    (q5) edge [bend right=45] node {$\epsilon$} (q0)
    ;
    
  \end{tikzpicture}
\end{center}

%------------------------------------------------------------------------------
\newpage
Q2. Sipser 2.2.

\textsc{Solution}.

(a.) %{w| the length of w is at most 5} \dot {w| every odd position of w is a 1}
\begin{center}
  \begin{tikzpicture}[shorten >=1pt,>=triangle 60,double distance=2pt,node distance=2cm,auto,initial text=]

    \node[state,initial]   (q00)  at (0, 0)  {$q_00$};
    \node[state]           (q01)  at (2, 0)  {$q_01$};
    \node[state]           (q02)  at (4, 0)  {$q_02$};
    \node[state]           (q03)  at (6, 0)  {$q_03$};
    \node[state]           (q04)  at (8, 0)  {$q_04$};
    \node[state]           (q05)  at (10, 0) {$q_05$};
    \node[state]           (q06)  at (12, 0) {$q_06$};
    

    \node[state,accepting] (q10)  at (0, -4) {$q_10$};
    \node[state,accepting] (q11)  at (2, -4) {$q_11$};
    \node[state,accepting] (q12)  at (4, -4) {$q_12$};
    \node[state]           (q13)  at (6, -4) {$q_13$};

    \path[->]
    (q00) edge                 node {$0,1$}      (q01)
    (q00) edge [left]          node {$\epsilon$} (q10)
    
    (q01) edge                 node {$0,1$}      (q02)
    (q01) edge [bend right=15] node {$\epsilon$} (q10)
   
    (q02) edge                 node {$0,1$}      (q03)
    (q02) edge [bend right=15] node {$\epsilon$} (q10)
   
    (q03) edge                 node {$0,1$}      (q04)
    (q03) edge [bend right=15] node {$\epsilon$} (q10)
   
    (q04) edge                 node {$0,1$}      (q05)
    (q04) edge [bend right=15] node {$\epsilon$} (q10)
   
    (q05) edge                 node {$0,1$}      (q06)
    (q05) edge [bend right=15] node {$\epsilon$} (q10)

    (q06) edge [loop above]    node {$0,1$}      ()

    %-----------------------------------------------------
    (q10) edge                 node {$1$}        (q11)
    
    (q11) edge [bend left=15]  node {$0,1$}      (q12)
    
    (q12) edge                 node {$0$}        (q13)
    (q12) edge [bend left=15]  node {$1$}        (q11)
    
    (q13) edge [loop above]    node {$0,1$}      (q11)
    ;
    
  \end{tikzpicture}
\end{center}


(b.) %{w| w contains at least three 1s} \dot {}
\begin{center}
  \begin{tikzpicture}[shorten >=1pt,>=triangle 60,double distance=2pt,node distance=2cm,auto,initial text=]
    
    \node[state,initial]   (q0)  at (0, 0)  {$q_0$};
    \node[state]           (q1)  at (2, 0)  {$q_1$};
    \node[state]           (q2)  at (4, 0)  {$q_2$};
    \node[state]           (q3)  at (6, 0)  {$q_3$};
    \node[state,accepting] (q4)  at (8, 0)  {$q_4$};
    
    \path[->]
    (q0) edge [loop above]    node {$0$}        ()
    (q0) edge                 node {$1$}        (q1)

    (q1) edge [loop above]    node {$0$}        ()
    (q1) edge                 node {$1$}        (q2)

    (q2) edge [loop above]    node {$0$}        ()
    (q2) edge                 node {$1$}        (q3)

    (q3) edge [loop above]    node {$0$}        ()
    (q3) edge                 node {$\epsilon$} (q4)
    ;

  \end{tikzpicture}
\end{center}

%------------------------------------------------------------------------------
\newpage
Q3. Sipser 2.4.(b), (c), (e), (f)

\textsc{Solution}.

(a.) %{w| w contains at least three 1s}^{*}
\begin{center}
  \begin{tikzpicture}[shorten >=1pt,>=triangle 60,double distance=2pt,node distance=2cm,auto,initial text=]

    \node[state,initial,accepting] (q0)  at (0, 0)  {$q_0$};
    \node[state]                   (q1)  at (2, 0)  {$q_1$};
    \node[state]                   (q2)  at (4, 0)  {$q_2$};
    \node[state]                   (q3)  at (6, 0)  {$q_3$};
    \node[state,accepting]         (q4)  at (8, 0)  {$q_4$};

    \path[->]
    (q0) edge                 node {$\epsilon$} (q1)
    
    (q1) edge [loop above]    node {$0$}        ()
    (q1) edge                 node {$1$}        (q2)
    
    (q2) edge [loop above]    node {$0$}        ()
    (q2) edge                 node {$1$}        (q3)
    
    (q3) edge [loop above]    node {$0$}        ()
    (q3) edge                 node {$1$}        (q4)
    
    (q4) edge [loop above]    node {$0,1$}      ()
    (q4) edge [bend left=45]  node {$\epsilon$} (q1)
    ;
    
  \end{tikzpicture}
\end{center}

(b.) %{w| w contains at least two 0s and at most one 1}^{*}
\begin{center}
  \begin{tikzpicture}[shorten >=1pt,>=triangle 60,double distance=2pt,node distance=2cm,auto,initial text=]
    
    \node[state,initial,accepting] (q0)  at (0, 0)   {$q_0$};
    \node[state]                   (q1)  at (2, 0)   {$q_1$};
    \node[state]                   (q2)  at (4, 0)   {$q_2$};
    \node[state,accepting]         (q3)  at (6, 0)   {$q_3$};
    \node[state]                   (q4)  at (2, -2)  {$q_4$};
    \node[state]                   (q5)  at (4, -2)  {$q_5$};
    \node[state,accepting]         (q6)  at (6, -2)  {$q_6$};
    \node[state]                   (q7)  at (4, -4)  {$q_7$};
    
    \path[->]
    (q0) edge                  node {$\epsilon$} (q1)
    
    (q1) edge                  node {$0$}        (q2)
    (q1) edge [left]           node {$1$}        (q4)
    
    (q2) edge                  node {$0$}        (q3)
    (q2) edge [left]           node {$1$}        (q5)
    
    (q3) edge [loop above]     node {$0$}        ()
    (q3) edge [left]           node {$1$}        (q6)
    (q3) edge [bend right=45]  node {$\epsilon$} (q1)
    
    (q4) edge                  node {$0$}        (q5)
    (q4) edge [bend right=15]  node {$1$}        (q7)
    
    (q5) edge                  node {$0$}        (q6)
    (q5) edge [left]           node {$1$}        (q7)
    
    (q6) edge [loop right]     node {$0$}        ()
    (q6) edge [bend left=15]   node {$1$}        (q7)
    (q6) edge [bend right=15]  node {$\epsilon$} (q1)
    
    (q7) edge [loop below]     node {$0,1$}      ()
    
    ;
    
  \end{tikzpicture}
\end{center}

(c.) %{}^{*}
\begin{center}
  \begin{tikzpicture}[shorten >=1pt,>=triangle 60,double distance=2pt,node distance=2cm,auto,initial text=]
    
    \node[state,initial,accepting] (q0)  at (0, 0)  {$q_0$};
    \node[state,accepting]         (q1)  at (2, 0)  {$q_1$};
    
    \path[->]
    (q0) edge                  node {$\epsilon$} (q1)
    (q1) edge [loop above]     node {$\epsilon$} ()
    ;

  \end{tikzpicture}
\end{center}

%------------------------------------------------------------------------------
\newpage
Q4. Sipser 2.5. For each questions, draw PDA state diagram and
informally describe how it words.
You do \textit{not} need to give the formal definition.

\textsc{Solution}.

Regular Expression: $D = \{\beta(\beta\beta)^*(\alpha\alpha)^*\}$

\begin{center}
  \begin{tikzpicture}[shorten >=1pt,>=triangle 60,double distance=2pt,node distance=2cm,auto,initial text=]
    
    \node[state,initial]   (q0) at (0,0) {$q_0$};
    \node[state,accepting] (q1) at (2,0) {$q_1$};
    \node[state]           (q2) at (4,0) {$q_2$};
    \node[state,accepting] (q3) at (6,0) {$q_3$};
    \node[state]           (q4) at (8,0) {$q_4$};
    
    \path[->]
    (q0) edge [bend right=45] node {$\alpha$}       (q4)
    (q0) edge [bend left=15]  node {$\beta$}        (q1)
    
    (q1) edge                 node {$\alpha$}       (q2)
    (q1) edge [bend left=15]  node {$\beta$}        (q0)
    
    (q2) edge [bend left=15]  node {$\alpha$}       (q3)
    (q2) edge [bend right=45] node {$\beta$}        (q4)
    
    (q3) edge [bend left=15]  node {$\alpha$}       (q2)
    (q3) edge                 node {$\beta$}        (q4)
    
    (q4) edge [loop right]    node {$\alpha,\beta$} ()
    ;
    
  \end{tikzpicture}
\end{center}

%------------------------------------------------------------------------------
\newpage
Q5. Sipser 2.6 (b) and (d).

\textsc{Solution}.

Regular Expression: $D = \{\beta(\beta\beta)^*(\alpha\alpha)^*\}$

\begin{center}
  \begin{tikzpicture}[shorten >=1pt,>=triangle 60,double distance=2pt,node distance=2cm,auto,initial text=]
    
    \node[state,initial]   (q0) at (0,0) {$q_0$};
    \node[state,accepting] (q1) at (2,0) {$q_1$};
    \node[state]           (q2) at (4,0) {$q_2$};
    \node[state,accepting] (q3) at (6,0) {$q_3$};
    \node[state]           (q4) at (8,0) {$q_4$};
    
    \path[->]
    (q0) edge [bend right=45] node {$\alpha$}       (q4)
    (q0) edge [bend left=15]  node {$\beta$}        (q1)
    
    (q1) edge                 node {$\alpha$}       (q2)
    (q1) edge [bend left=15]  node {$\beta$}        (q0)
    
    (q2) edge [bend left=15]  node {$\alpha$}       (q3)
    (q2) edge [bend right=45] node {$\beta$}        (q4)
    
    (q3) edge [bend left=15]  node {$\alpha$}       (q2)
    (q3) edge                 node {$\beta$}        (q4)
    
    (q4) edge [loop right]    node {$\alpha,\beta$} ()
    ;
    
  \end{tikzpicture}
\end{center}

%------------------------------------------------------------------------------
\newpage
Q6. Sipser 2.9.

\textsc{Solution}.

\begin{center}
  \begin{tikzpicture}[shorten >=1pt,>=triangle 60,double distance=2pt,node distance=2cm,auto,initial text=]

    \node[state,initial,accepting] (q0)  at (0, 0)  {$q_0$};
    \node[state,accepting]         (q1)  at (2, 0)  {$q_1$};
    \node[state,accepting]         (q2)  at (4, 0)  {$q_2$};
    \node[state,accepting]         (q3)  at (2, -2) {$q_3$};
    \node[state]                   (q4)  at (4, -2) {$q_4$};

    \path[->]
    (q0) edge [loop above]    node {$0$}   ()
    (q0) edge                 node {$1$}   (q1)
    
    (q1) edge [bend left=15]  node {$0$}   (q2)
    (q1) edge [left]          node {$1$}   (q3)
    
    (q2) edge [bend left=15]  node {$0$}   (q1)
    (q2) edge [left]          node {$1$}   (q4)
    
    (q3) edge [loop below]    node {$0$}   ()
    (q3) edge                 node {$1$}   (q4)
    
    (q4) edge [loop below]    node {$0,1$} ()
    ;
    
  \end{tikzpicture}
\end{center}


%------------------------------------------------------------------------------
\newpage
Q7. Sipser 2.10. (Hint: Look at the word \lq\lq or".)

\textsc{Solution}.

\input{q07.tex}

%------------------------------------------------------------------------------
\newpage
Q8. Sipser 2.13.

\textsc{Solution}.


(a) \textsc{Solution provided}.
\begin{align*}
  \ep, a   &\in      L(a^*b^*) \\
  ba, baa  &\not\in  L(a^*b^*) 
\end{align*}

(b) \textsc{Solution provided}.
\begin{align*}
  ab, abab   &\in     L(a(ba)^*b) \\
  b, b       &\not\in L(a(ba)^*b) 
\end{align*}

(c) \textsc{Solution provided}.
\begin{align*}
  \ep, a     &\in     L(a^* \cup b^*) \\
  ab, ba     &\not\in L(a^* \cup b^*) 
\end{align*}

(d)

(e)

(f)

(g)

(h)



\end{document}

\input{myassignmentpreamble}
\input{ciss362}
\input{yliow}

\renewcommand\AUTHOR{John Doe}
\renewcommand\EMAIL{jdoe@cougars.ccis.edu}
\renewcommand\TITLE{Assignment 2}

\begin{document}
\topmatter

You should study Sipser chapter 0 and do all the problems in the Exercises
and Problems section.
Form study groups. Meet at CS Hangout. Compare solutions.
As always I very strongly encourage discussion.
But you must must write your own work and you must understand what you wrote
down.
If I call on you to explain your work and you can't, then that's considered
plaigiarism.

For Q1, enter your work in \verb!q01.tex!.
Several examples are included so that you and modify the given \LaTeX\ code.
You can also find some information on \LaTeX\ at my website
\url{http://yliow.github.io} (go to the Tutorial section and look for
the latex pdf.) Or just ask me.

\newpage
\textsc{Proving $P(n)$ for all $n \geq n_0$ where $n_0$ is a fixed integer}

Recall that if you want to prove
$P(n)$ is true for all $n \geq 42$,
you have to do two things:
\begin{tightlist}
  \li Prove $P(42)$ is true
  \li Let $n \geq 42$. Assume $P(n)$ holds and prove $P(n + 1)$ holds as well.
\end{tightlist}
If you can achieve the above two points, then you can claim
\begin{tightlist}
  \li $P(n)$ is true for all $n \geq 42$
\end{tightlist}
This is one form of mathematical induction called \textbf{weak mathematical induction}.
The \textbf{strong mathematical induction} says that if you can do two things:
\begin{tightlist}
  \li Prove $P(42)$
  \li Let $n \geq 42$. Assume $P(42), P(42 + 1), ..., P(n)$ holds and prove $P(n + 1)$ holds as well.
\end{tightlist}
If you achieve the above two points, then you can claim
\begin{tightlist}
  \li $P(n)$ is true for all $n \geq 42$
\end{tightlist}
The \lq\lq42" above can be replaced by any integer (including some negative integer).

The above two induction techniques proves $P(n)$ are all true for $n \geq n_0$.
Mathematical induction also allows you go \lq\lq go backward".
If you can do the following:
\begin{tightlist}
  \li Prove $P(n_0)$ is true
  \li Let $n \leq n_0$. Assume $P(n)$ holds and prove $P(n - 1)$ holds as well.
\end{tightlist}
then you can claim
\begin{tightlist}
  \li $P(n)$ is true for all $n \leq n_0$
\end{tightlist}
This is also called weak mathematical induction.
The strong induction going backward holds too.
If you can achieve the following:
\begin{tightlist}
  \li Prove $P(n_0)$ is true
  \li Let $n \leq n_0$. Assume $P(n_0), P(n_0 - 1), ..., P(n)$  holds and prove $P(n - 1)$ holds as well.
\end{tightlist}
then you can claim
\begin{tightlist}
  \li $P(n)$ is true for all $n \leq n_0$
\end{tightlist}
So induction allows you to prove $P(n)$ when $n$ goes to infinity or
when $n$ goes to negative infinity.

There is another method that's important and does the same thing for you.

Sometimes, it is possible to prove $P(n)$ for all $n \geq 1$ by using the
\lq\lq proof by contradiction" method by using
the well-ordering principle (WOP) which says that
\begin{tightlist}
  \li Every nonempty subset of $\N = \{0, 1, 2, ...\}$ has a least element,
  i.e., if $X$ is a nonempty subset of $\N$, then there is some
  $m \in X$ such that $m \leq x$ for all $x \in X$.
\end{tightlist}
(See class notes.)
In this case, frequently, the proof involves an argument of the form:
\lq\lq Suppose it's not true that $P(n)$ holds for all $n \geq n_0$".
Then the set
\[
X = \{ n \mid n \geq n_0, \,\,\, P(n) \text{ does not hold}\}
\]
is a nonempty subset of $\N$.
By WOP, $X$ has a least element, i.e.,
there is a smallest $m$ such that $m \geq n_0$ and $P(m)$ is false.
And you continue to prove that something goes wrong,
i.e.,you attempt to arrive at a contradiction.
This is frequently done in one of two ways.
Here's one way to achieve this:
Since $m$ is the least element of $X$,
all $n \in \N$ with $n < m$ must satisfy $P(n)$.
From such $n$, you then show that in fact $P(m)$ holds, which
clearly is a contradiction.
The second method is this:
Try to find some $m' < m$
such that $P(m')$ is also false.
This would contradict the fact that $m$ is the least elment of
$X$.

(\textsc{Aside.}
WOP is related to the proof method called \textbf{Fermat's infinite descent}.
Applying the same argument above on $k'$ and assuming $P(k')$ is false,
you would arrive at another
$k'' < k'$ such that $P(k'')$ is false, etc.
This gives you infinitely many positive integers
$k > k' > k'' > k''' > \cdots$.
This is clearly impossible since there can only be finitely many positive
integer from $n_0$ up to $k$.)

Until you know how to write induction proofs properly, you must follow these
instructions for writing a proof that uses induction.
\begin{tightlist}
\item[1.]
  Paragraph 1: State you $P(n)$ and the range of values for $n$.
  If a problem involves proving multiple statements, you can also use
  $Q(n)$, $R(n)$, etc.
  State what method you are using (weak or strong induction).
  The default is weak induction, i.e., if you want you are using
  mathematical induction, it means you are using weak mathematical induction.
\item [2.]
  Paragraph 2: State you are proving the base case.
  Prove the base case.
\item [3.]
  Paragraph 3: State you are proving the inductive case.
  State your inductive hypothesis and state what you are going to prove.
  Then prove it. If the proof is long, state what you have proven.
  It's even a good idea to state it anyway. That's called good writing:
  State at the beginning of a paragraph what you want to do, do it, then
  remind the reader the goal at the beginning of the paragraph.
  (This is the longest part of the proof. If necessary, you might need
  more than one paragraph.)
\item [4.]
  Paragraph 4:
  State, quoting the method (i.e., induction), what you have proven.
\end{tightlist}
Once you are done with writing about 50 induction proofs, you can use a
freer form.

Here are some examples:

\newpage

\begin{thm}
  If $n \geq 0$, then
  \[
    1 + 2 + \cdots + n = \frac{n(n + 1)}{2}
  \]
  (Note that if $n = 0$, then the expression on the left-hand side of the above
  equation is 0 by definition.
  In other words an empty sum -- sum of no terms -- is defined to be 0.)
\end{thm}

\textit{Proof}.
We will prove this by weak mathematical induction.
For $n \geq 0$, we define
\[
  P(n) = \biggl( 1 + 2 + \cdots + n = \frac{n(n + 1)}{2} \biggr)
\]

\textsc{Base case.}
When $n = 0$, we have
\[
1 + 2 + \cdots + n = 0 = \frac{0(0 + 1)}{2} = \frac{n (n + 1)}{2}
\]
Hence $P(0)$ holds.

\textsc{Inductive case.}
Assume $P(n)$ holds where $n \geq 0$, i.e., we assume
\[
  P(n) = \biggl( 1 + 2 + \cdots n = \frac{n(n + 1)}{2} \biggr)
\]
holds.
We want to show $P(n + 1)$ holds, i.e., we want to show
\[
P(n + 1) =
\biggl(
1 + 2 + \cdots n + 1 = \frac{(n + 1)(n + 1 + 1)}{2}
\biggr)
\]
is true.
Since $P(n)$ holds, we have
\begin{align*}
  1 + 2 + \cdots n
  &= \frac{n(n + 1)}{2} \\
  \THEREFORE 1 + 2 + \cdots n + (n + 1)
  &= \frac{n(n + 1)}{2} + (n + 1) \\
  &= \frac{n(n + 1) + 2(n + 1)}{2} \\
  &= \frac{(n + 1)(n + 2)}{2} \\
  &= \frac{(n + 1)((n + 1) + 1)}{2}
\end{align*}
i.e., $P(n + 1)$ holds.

Therefore, by weak mathematical induction,
$P(n)$ holds for all $n \geq 0$, i.e.,
for all $n \geq 0$,
\[
  1 + 2 + \cdots + n = \frac{n(n + 1)}{2}
\]
\qed




\newpage
\begin{thm}
  If $n \geq 0$, then
  \[
    1 + 2 + \cdots + n = \frac{n(n + 1)}{2}
  \]
\end{thm}

\textit{Proof}.
We will prove this using the well-ordering principle.
Let $P(n)$ be the statement
\[
    P(n) = \biggl( 1 + 2 + \cdots + n = \frac{n(n + 1)}{2} \biggr)
\]
Suppose on the contrary that $P(n)$ does not hold for all $n \geq 0$.
Let $n_0 \in \N = \{0, 1, 2, ...\}$ such that $P(n_0)$ does not hold.
Note that by definition when $n = 0$,
\[
1 + 2 + \cdots + n = 0 = \frac{0(0 + 1)}{2} = \frac{n(n + 1)}{2}
\]
Hence the statement $P(0)$ holds.
Therefore $n_0 > 0$.

Let
\[
X = \{n \mid n \geq 0, \,\,\, P(n) \text{ does not hold}\}
\]
Note that $X \subseteq \N = \{0, 1, 2, ...\}$.
Since $n_0 \in X$, $X$ is a nonempty subset of $\N = \{0, 1, 2, ...\}$.
Therefore by the well-ordering principle, $X$ has a least element, say $m$.
Since $m \in X$, we have 
\[
1 + 2 + \cdots + m \neq \frac{m(m + 1)}{2} \tag{a}
\]
i.e., $P(m)$ does not hold.
Since $P(0)$ holds, we have $m > 0$ and hence $m - 1 \geq 0$.
Since $m$ is the least element of $X$, $m - 1 \geq 0$ is in
$\N = \{0, 1, 2, ...\}$, we have $m - 1 \notin X$.
Therefore $P(m - 1)$ holds, i.e.,
\begin{align*}
             1 + 2 + \cdots + (m - 1)     &= \frac{(m - 1)((m - 1) + 1)}{2} \\
  \THEREFORE 1 + 2 + \cdots + (m - 1) + m &= \frac{(m - 1)((m - 1) + 1)}{2} + m \\
                                          &= \frac{(m - 1)m}{2} + m \\
                                          &= \frac{(m - 1)m + 2m}{2} \\
                                          &= \frac{m^2 - m + 2m}{2} \\
                                          &= \frac{m^2 + m}{2} \\
                                          &= \frac{m(m + 1)}{2} \\
\end{align*}
Hence $P(m)$ holds.
This contradicts (a).
Therefore $X$ must be empty.
Hence
\[
1 + 2 + \cdots + n = \frac{n(n + 1)}{2}
\]
for all $n \geq 0$.
\qed

\newpage
\begin{thm}
  Let $T$ be a tree wth at least one node,
  i.e., a connected simple graph with at least one node and no cycles.
  Then $e = v - 1$ where $e$ and $v$ are the number of edges and
  nodes of $T$ respectively.
\end{thm}

(Note: A simple graph is a graph with no loops, i.e., no edge
joining a node to itself and no multi-edges, i.e., no multiple edges
joining the same two nodes.
A graph is connected is for every pair of distinct vertices $x,y$,
there is a path of edges from $x$ to $y$.)

\textit{Proof}.
For a graph $G$, we will write $v_G$ and $e_G$ for the number of
nodes and number of edges of $G$ (respectively).
For $n \geq 0$, we define the proposition $P(n)$ as follows:
\[
  P(n) = 
  (
  \text{If $T$ is a tree with at least one node
  \underline{and with $n$ edges}, then $e_T = v_T - 1$}
  )
\]
We will prove $P(n)$ holds for all $n \geq 0$ by strong mathematical induction.

\textsc{Base case}.
We will prove $P(0)$, i.e., we will prove that
if $T$ is a tree with at least one node and 
$0$ edges (i.e., $e_T = 0$), then $e_T = v_T - 1$.
If $T$ has at least two distinct nodes,
say $x$ and $y$, then $x$ and $y$ are not adjacent since there are
no edges in $T$.
But $T$ is connected.
This is a contradiction.
Hence $T$ has exactly one node, i.e., $v_T = 1$.
Therefore
\[
e_T = 0 = 1 - 1 = v_T - 1
\]
i.e., $P(0)$ holds.

\textsc{Inductive case}.
Let $n \geq 0$.
Assume $P(0), P(1), ..., P(n)$ holds.
We will show that $P(n + 1)$ holds, i.e., we will show that
if $T$ is a tree with at least one node and has $n + 1$ edges, then
\[
  e_T = v_T - 1
\]
Let $T$ be a tree with at least one node and $n + 1$ edges.
Since $n \geq 0$, we have $n + 1 \geq 1$.
Hence there is at least one edge in $T$, say
$e$ denote an edge in $T$ joining node $x$ and node $y$.
Since $T$ is simple, $x \neq y$.
Construct the graph
\[
  G = T - e
\]
i.e., $G$ is the graph $T$ with edge $e$ removed.
We have
\begin{align*}
  v_G &= v_T \\
  e_G &= e_T - 1 
\end{align*}
Note that $G$ contains all the nodes of $T$ and in particular
contains $x$ and $y$.
We claim that $G$ is made up of two disjoint trees.

First, we show that $G$ has exactly two connected components, i.e.,
$G$ is made up of two maximal connected subgraphs.
Suppose $k$ is the number of connected components of $G$.
Let $G_1, G_2, ..., G_k$ denote the connected components of $G$.
There are no paths joining a node in $G_i$ to a node in $G_j$ if $i \neq j$.
We will show $k = 2$ and furthermore each $G_i$ is a tree.

The nodes $x$ and $y$ are in $G$.
Therefore $x$ is in some $G_i$ and $y$ is in some $G_j$.
Note that $i \neq j$.
Otherwise, if $i = j$, $x$ and $y$ are in the same connected component
$G_i$ which implies that there is some path $p$ in $G_i$ joining $x$ and $y$.
Since $G_i$ is a subgraph of $G = T - e$ and $G$ does not contain edge $e$,
$G_i$ cannot contain $e$.
The path $p$ (which is in $G_i$) therefore also cannot contain edge $e$.
Hence $p$ and edge $e$ will form a cycle in $T$.
This is a contradiction since $T$ is a tree and cannot have a cycle.
Hence we have shown that $i \neq j$, i.e., $x$ and $y$ are in two different
connnected components of $G$.

Now suppose $k > 2$, i.e.,
suppose there are three connected components.
Recall that $x$ is in some $G_i$ and $y$ is in some $G_j$
with $i \neq j$.
Since there are at least three connected components, there is some
$k$
such that $k \neq i, k \neq j$. 
Let $z$ be a node in $G_k$.
Since $T$ is a tree, there is a path $p$ in $T$ from $x$ to $z$.
There are no repeated nodes in $p$. 
Since $x$ and $z$ are in different connected components,
the path $p$ must leave $G_i$ and must enter $G_k$.
The only edge that leaves $G_i$ is the edge $e$.
Since $y$ is in $G_j$,
path $p$ will leave $G_i$ and enter $G_j$.
Therefore on entering $G_j$, $p$ contains $x$ and $y$.
However to arrive at $G_k$, which is not $G_j$,
the path $p$ has to leave $G_j$.
The only edge leaving $G_j$ is $e$, which means that
the path $p$ on leaving $G_k$ will repeat $x$.
This is a contradiction.
We conclude that $k = 2$, i.e., there are two
connected components in $G = T - e$.

We have now shown that there are two connected components $G_1, G_2$ in $G$.

We now show that the connected components $G_1, G_2$ in $G$
are trees.
Suppose there is a cycle $C$ in $G_1$.
Since $G_1$ is in $G = T - e$, the cycle $C$ is also in $G$
and is therefore in $T$.
This is a contradiction since $T$ is a tree.
(This shows that every subgraph of a tree cannot have cycles.)

We have now shown that $T - e$ is made up of two
disjoint trees, say $T_1$ and $T_2$.

Since $T_1, T_2$ are both subgraphs of $G$ and $G$ is a subgraph of $T$, 
\[
  e_{T_i} \leq e_G = e_T  - 1 < e_T  = n + 1
\]
for $i = 1, 2$.
Therefore by induction hypothesis,
\begin{align*}
    e_{T_1} &= v_{T_1} - 1 \\
    e_{T_2} &= v_{T_2} - 1
\end{align*}
Adding these two equations, we get
\begin{align*}
    e_{T_1} + e_{T_2} &= v_{T_1} - 1 + v_{T_2} - 1 \\
    \THEREFORE e_{T_1} + e_{T_2} &= v_{T_1} + v_{T_2} - 2 \tag{a}
\end{align*}
Now note that since $T = G - e$,
the only difference between $T$ and $G$ is an edge.
Hence
\begin{align*}
  e_T &= e_G + 1 = e_{T_1} + e_{T_2} + 1 \tag{b}  \\
  v_T &= v_G = v_{T_1} + v_{T_2} \tag{c}
\end{align*}
Hence equations (a), (b), (c) gives us
\begin{align*}
  e_T
  &= e_{T_1} + e_{T_2} + 1 \\
  &= \left( v_{T_1} - 1 \right) + \left( v_{T_2} - 1 \right) + 1 \\
  &= v_{T_1} + v_{T_2} - 1 \\
  &= v_T - 1
\end{align*}
We have now shown $P(n + 1)$ holds if $P(0), P(1), ..., P(n)$ hold.

Therefore,
by strong mathematical induction, $P(n)$ holds for all $n \geq 0$, i.e.,
we have shown that if $T$ is a tree with at least one node then
\[
  e_T = v_T - 1
\]
\qed


\newpage\textsc{Template for induction proofs}

\textbf{A. Here is the template for weak induction proofs:}

We will prove the above statemet by weak induction.
For $n \geq ?$, let $P(n)$ be the statements
\[
  P(n) = \biggl( n^2 \text{ is a prime} \biggr)
\]

\textsc{Base case}.
We now prove the base case $P(?)$. [... your proof ...]
Hence $P(?)$ holds.

\textsc{Inductive case}.
We now prove the inductive case.
Assume $P(n)$ holds where $n \geq ?$. 
[... your proof ...]
Hence $P(n + 1)$ holds.

Therefore, by weak mathematical induction,
$P(n)$ is true for all $n \geq ?$, i.e.,
for any $n \geq ?$,
\[
  n^2 \text{ is a prime}
\]
    

\textbf{B. Here is the template for strong induction proofs:}

We will prove $P(n)$ is true for all $n \geq ?$ using
strong mathematical induction.
For $n \geq ?$, let $P(n)$ be the statements
\[
  P(n) = \biggl( n^2 \text{ is a prime} \biggr)
\]

\textsc{Base case}.
We now prove the base case $P(?)$. [... your proof ...]
Hence $P(?)$ holds.

\textsc{Inductive case}.
We now prove the inductive case.
Assume $P(?), P(?+1), ..., P(n)$ hold where $n \geq ?$. 
[...]
Hence $P(n + 1)$ holds.

Therefore, by strong mathematical induction,
$P(n)$ is true for all $n \geq ?$, i.e.,
for any $n \geq ?$,
\[
  n^2 \text{ is a prime}
\]

\textbf{B. Here is a template for WOP proofs:}

We will prove the above statement using the well-ordering principle.
For $n \geq ?$, let $P(n)$ be 
\[
  P(n) = \biggl( n^2 \text{ is a prime} \biggr)
\]
Assume on the contrary that $P(n)$ does not hold for all $n \geq ?$.
Then there is some $n$ such that $P(n)$ does not hold.
Define
\[
X = \{ n \geq ? \mid P(n) \text{ does not hold} \}
\]
[... now prove that $X$ is a nonempty subset of $\Z$ that is bounded below,
  or a nonempty subset of $\N$.]

By the well-ordering principle $X$ has a least element, say $m$.
Since $m$ is the least element of $X$,
$P(n)$ holds for all $? \leq n < m$.
[Now derive a contradiction.
  Warning: You usually have to use some $n$ such that 
  $? \leq n < m$.
  This means that you have to explain why $m > ?$.
]
This is a contradiction because [...]

Therefore, $P(n)$ holds for all $n \geq ?$.



%------------------------------------------------------------------------------
\newpage
Q1. Sipser 0.10. 

\SOLUTION

(a.) %{w| w begins with a 1 and ends with a 0} \cup {w| w contains at least three 1s}
\begin{center}
  \begin{tikzpicture}[shorten >=1pt,>=triangle 60,double distance=2pt,node distance=2cm,auto,initial text=]

    \node[state,initial]   (q0)  at (0, 0)  {$q_0$};
    
    \node[state]           (q00)  at (2, 2)  {$q_00$};
    \node[state]           (q01)  at (4, 2)  {$q_01$};
    \node[state,accepting] (q02)  at (6, 2)  {$q_02$};
    
    \node[state]           (q10)  at (2, -2)  {$q_10$};
    \node[state]           (q11)  at (4, -2)  {$q_11$};
    \node[state]           (q12)  at (6, -2)  {$q_12$};
    \node[state,accepting] (q13)  at (8, -2)  {$q_13$};

    \path[->]
    (q0)  edge [bend left=30]  node {$\epsilon $} (q00)
    (q0)  edge [bend right=30] node {$\epsilon $} (q10)

    %--------------------------------------------------------    
    (q00) edge              node {$1$}      (q01)
   
    (q01) edge [bend left=15] node {$0$}    (q02)
    (q01) edge [loop above]   node {$1$}    ()
   
    (q02) edge [loop above]   node {$0$}    ()
    (q02) edge [bend left=15] node {$1$}    (q01)

    %--------------------------------------------------------    
    (q10) edge [loop above]   node {$0$}    ()
    (q10) edge                node {$1$}    (q11)
   
    (q11) edge [loop above]   node {$0$}    ()
    (q11) edge                node {$1$}    (q12)
   
    (q12) edge [loop above]   node {$0$}    ()
    (q12) edge                node {$1$}    (q13)

    (q13) edge [loop above]   node {$0,1$} ()
    ;
    
  \end{tikzpicture}
\end{center}

(b.) %{w| w contains the substring 0101 (i.e., w = x0101y for some x and y)} \cup {w| w doesn’t contain the substring 110}
\begin{center}
  \begin{tikzpicture}[shorten >=1pt,>=triangle 60,double distance=2pt,node distance=2cm,auto,initial text=]

    \node[state,initial]   (q0)  at (0, 0)  {$q_0$};
    
    \node[state]           (q00)  at (0, 2)  {$q_00$};
    \node[state]           (q01)  at (2, 2)  {$q_01$};
    \node[state,accepting] (q02)  at (4, 2)  {$q_02$};
    \node[state,accepting] (q03)  at (6, 2)  {$q_03$};
    \node[state,accepting] (q04)  at (8, 2)  {$q_04$};
    
    \node[state,accepting] (q10)  at (0, -2)  {$q_10$};
    \node[state,accepting] (q11)  at (2, -2)  {$q_11$};
    \node[state,accepting] (q12)  at (4, -2)  {$q_12$};
    \node[state]           (q13)  at (6, -2)  {$q_13$};

    \path[->]
    (q0)  edge [left] node {$\epsilon $} (q00)
    (q0)  edge [left] node {$\epsilon $} (q10)

    %--------------------------------------------------------    
    (q00) edge                 node {$0$}   (q01)
    (q00) edge [loop above]    node {$1$}   ()
    
    (q01) edge                 node {$0$}   (q02)
    (q01) edge [loop above]    node {$0$}   ()
    
    (q02) edge                 node {$0$}   (q03)
    (q02) edge [bend left=45]  node {$1$}   (q00)
    
    (q03) edge [bend left=45]  node {$0$}   (q01)
    (q03) edge                 node {$1$}   (q04)
    
    (q04) edge [loop above]    node {$0,1$} ()
    %--------------------------------------------------------    
    (q10) edge [loop below]    node {$0$}   ()
    (q10) edge                 node {$1$}   (q11)
    
    (q11) edge [bend left=30]  node {$0$}   (q10)
    (q11) edge                 node {$1$}   (q12)
    
    (q12) edge                 node {$0$}   (q13)
    (q12) edge [loop above]    node {$1$}   ()

    (q13) edge [loop above]    node {$0,1$} ()
    ;

  \end{tikzpicture}
\end{center}

%------------------------------------------------------------------------------
\newpage
Q2. Sipser 0.12.

\SOLUTION

(a.) %{w| the length of w is at most 5} \dot {w| every odd position of w is a 1}
\begin{center}
  \begin{tikzpicture}[shorten >=1pt,>=triangle 60,double distance=2pt,node distance=2cm,auto,initial text=]

    \node[state,initial]   (q00)  at (0, 0)  {$q_00$};
    \node[state]           (q01)  at (2, 0)  {$q_01$};
    \node[state]           (q02)  at (4, 0)  {$q_02$};
    \node[state]           (q03)  at (6, 0)  {$q_03$};
    \node[state]           (q04)  at (8, 0)  {$q_04$};
    \node[state]           (q05)  at (10, 0) {$q_05$};
    \node[state]           (q06)  at (12, 0) {$q_06$};
    

    \node[state,accepting] (q10)  at (0, -4) {$q_10$};
    \node[state,accepting] (q11)  at (2, -4) {$q_11$};
    \node[state,accepting] (q12)  at (4, -4) {$q_12$};
    \node[state]           (q13)  at (6, -4) {$q_13$};

    \path[->]
    (q00) edge                 node {$0,1$}      (q01)
    (q00) edge [left]          node {$\epsilon$} (q10)
    
    (q01) edge                 node {$0,1$}      (q02)
    (q01) edge [bend right=15] node {$\epsilon$} (q10)
   
    (q02) edge                 node {$0,1$}      (q03)
    (q02) edge [bend right=15] node {$\epsilon$} (q10)
   
    (q03) edge                 node {$0,1$}      (q04)
    (q03) edge [bend right=15] node {$\epsilon$} (q10)
   
    (q04) edge                 node {$0,1$}      (q05)
    (q04) edge [bend right=15] node {$\epsilon$} (q10)
   
    (q05) edge                 node {$0,1$}      (q06)
    (q05) edge [bend right=15] node {$\epsilon$} (q10)

    (q06) edge [loop above]    node {$0,1$}      ()

    %-----------------------------------------------------
    (q10) edge                 node {$1$}        (q11)
    
    (q11) edge [bend left=15]  node {$0,1$}      (q12)
    
    (q12) edge                 node {$0$}        (q13)
    (q12) edge [bend left=15]  node {$1$}        (q11)
    
    (q13) edge [loop above]    node {$0,1$}      (q11)
    ;
    
  \end{tikzpicture}
\end{center}


(b.) %{w| w contains at least three 1s} \dot {}
\begin{center}
  \begin{tikzpicture}[shorten >=1pt,>=triangle 60,double distance=2pt,node distance=2cm,auto,initial text=]
    
    \node[state,initial]   (q0)  at (0, 0)  {$q_0$};
    \node[state]           (q1)  at (2, 0)  {$q_1$};
    \node[state]           (q2)  at (4, 0)  {$q_2$};
    \node[state]           (q3)  at (6, 0)  {$q_3$};
    \node[state,accepting] (q4)  at (8, 0)  {$q_4$};
    
    \path[->]
    (q0) edge [loop above]    node {$0$}        ()
    (q0) edge                 node {$1$}        (q1)

    (q1) edge [loop above]    node {$0$}        ()
    (q1) edge                 node {$1$}        (q2)

    (q2) edge [loop above]    node {$0$}        ()
    (q2) edge                 node {$1$}        (q3)

    (q3) edge [loop above]    node {$0$}        ()
    (q3) edge                 node {$\epsilon$} (q4)
    ;

  \end{tikzpicture}
\end{center}

%------------------------------------------------------------------------------
\newpage
Q3.
Using mathematical induction, give a complete proof of the following fact:
\[
  4(1^3 + 2^3 + \cdots + n^3) = n^2(n+1)^2
\]
for $n \geq 0$.

\SOLUTION


(b)
\begin{center}
  \begin{tikzpicture}[shorten >=1pt,>=triangle 60,double distance=2pt,node distance=2cm,auto,initial text=]
    
    \node[state,initial]   (q0) at (0, 0) {$q_0$};
    \node[state]           (q1) at (2, 0) {$q_0$};
    \node[state]           (q2) at (4, 0) {$q_0$};
    \node[state]           (q3) at (6, 0) {$q_0$};
    \node[state,accepting] (q4) at (8, 0) {$q_0$};

    \path[->]
    (q0) edge                 node {$0$}   (q1)
    (q0) edge [loop above]    node {$1$}   ()
    
    (q1) edge [bend left=45]  node {$0$}   (q0)
    (q1) edge                 node {$1$}   (q2)
    
    (q2) edge                 node {$0$}   (q3)
    (q2) edge [bend right=45] node {$1$}   (q0)
    
    (q3) edge [bend left=45]  node {$0$}   (q0)
    (q3) edge                 node {$1$}   (q4)
    
    (q4) edge [loop right]    node {$1,0$} ()
    ;
  \end{tikzpicture}
\end{center}

(c)
\begin{center}
  \begin{tikzpicture}[shorten >=1pt,>=triangle 60,double distance=2pt,node distance=2cm,auto,initial text=]
    
    \node[state,initial,accepting] (q0) at (0, 0)  {$q_0$};
    \node[state]                   (q1) at (0, -4) {$q_1$};
    \node[state,accepting]         (q2) at (3, 0)  {$q_2$};
    \node[state]                   (q3) at (3, -4) {$q_3$};
    \node[state,accepting]         (q4) at (6, 0)  {$q_4$};
    \node[state,accepting]         (q5) at (6, -4) {$q_5$};
    
    \path[->]
    (q0) edge [bend left=15]          node {$0$}   (q1)
    (q0) edge                         node {$1$}   (q2)

    (q1) edge [bend left=15]          node {$0$}   (q0)
    (q1) edge                         node {$1$}   (q3)

    (q2) edge [bend left=15]          node {$0$}   (q3)
    (q2) edge                         node {$1$}   (q4)

    (q3) edge [bend left=15]          node {$0$}   (q2)
    (q3) edge                         node {$1$}   (q5)
    
    (q4) edge [bend left=15]          node {$0$}   (q5)
    
    (q5) edge [bend left=15]          node {$0$}   (q4)
    ;
  \end{tikzpicture}
\end{center}

(d)
\begin{center}
  \begin{tikzpicture}[shorten >=1pt,>=triangle 60,double distance=2pt,node distance=2cm,auto,initial text=]
    
    \node[state,initial]   (q0) at (0, 0) {$q_0$};
    \node[state,accepting] (q1) at (4, 0) {$q_1$};

    \path[->]
    (q0) edge node {$0$}   (q1)
    ;
  \end{tikzpicture}
\end{center}



(e)
\begin{center}
  \begin{tikzpicture}[shorten >=1pt,>=triangle 60,double distance=2pt,node distance=2cm,auto,initial text=]
    
    \node[state,initial]   (q0) at (0, 0) {$q_0$};
    \node[state,accepting] (q1) at (3, 0) {$q_1$};
    \node[state]           (q2) at (6, 0) {$q_2$};

    \path[->]
    (q0) edge                 node {$0$} (q1)
    (q0) edge [bend right=30] node {$1$} (q2)
    
    (q1) edge [loop above]    node {$0$} ()
    (q1) edge [bend left=15]  node {$1$} (q2)
    
    (q2) edge [bend left=15]  node {$0$} (q1)
    (q2) edge [loop right]    node {$1$} ()

    ;
  \end{tikzpicture}
\end{center}

\newpage
(g)
\begin{center}
  \begin{tikzpicture}[shorten >=1pt,>=triangle 60,double distance=2pt,node distance=2cm,auto,initial text=]
    
    \node[state,initial,accepting] (q0) at (0, 0) {$q_0$};

    \path[->]
    ;
  \end{tikzpicture}
\end{center}

(h)
\begin{center}
  \begin{tikzpicture}[shorten >=1pt,>=triangle 60,double distance=2pt,node distance=2cm,auto,initial text=]
    
    \node[state,initial,accepting] (q0) at (0, 0) {$q_0$};

    \path[->]
    (q0) edge [loop right] node {$0$} ()
    ;
  \end{tikzpicture}
\end{center}


%------------------------------------------------------------------------------
\newpage
Q4.
Using WOP, give a complete proof of the following fact:
\[
  4(1^3 + 2^3 + \cdots + n^3) = n^2(n+1)^2
\]
for $n \geq 0$.


\SOLUTION

Regular Expression: $D = \{\beta(\beta\beta)^*(\alpha\alpha)^*\}$

\begin{center}
  \begin{tikzpicture}[shorten >=1pt,>=triangle 60,double distance=2pt,node distance=2cm,auto,initial text=]
    
    \node[state,initial]   (q0) at (0,0) {$q_0$};
    \node[state,accepting] (q1) at (2,0) {$q_1$};
    \node[state]           (q2) at (4,0) {$q_2$};
    \node[state,accepting] (q3) at (6,0) {$q_3$};
    \node[state]           (q4) at (8,0) {$q_4$};
    
    \path[->]
    (q0) edge [bend right=45] node {$\alpha$}       (q4)
    (q0) edge [bend left=15]  node {$\beta$}        (q1)
    
    (q1) edge                 node {$\alpha$}       (q2)
    (q1) edge [bend left=15]  node {$\beta$}        (q0)
    
    (q2) edge [bend left=15]  node {$\alpha$}       (q3)
    (q2) edge [bend right=45] node {$\beta$}        (q4)
    
    (q3) edge [bend left=15]  node {$\alpha$}       (q2)
    (q3) edge                 node {$\beta$}        (q4)
    
    (q4) edge [loop right]    node {$\alpha,\beta$} ()
    ;
    
  \end{tikzpicture}
\end{center}

%------------------------------------------------------------------------------
\newpage
Q5. Write a complete proof of the Euclidean property of integers:
If $a$ and $b$ are integers such that $a \geq 0$ and $b > 0$,
then there exists integers $q$ and $r$ such that $q \geq 0$ and
\[
a = bq + r, \,\,\, 0 \leq r < b
\]
You must use WOP.
(I have given you the proof in class.
The point of this question is to study the proof in depth, write it up
properly in your own words, and edit your work until it's perfect.) 

\SOLUTION

\begin{center}
  \begin{tikzpicture}[shorten >=1pt,>=triangle 60,double distance=2pt,node distance=2cm,auto,initial text=]

    \node[state,initial,accepting] (q0)  at (0, 0)  {$q_0$};
    \node[state,accepting]         (q1)  at (2, 0)  {$q_1$};
    \node[state,accepting]         (q2)  at (4, 0)  {$q_2$};
    \node[state,accepting]         (q3)  at (2, -2) {$q_3$};
    \node[state]                   (q4)  at (4, -2) {$q_4$};

    \path[->]
    (q0) edge [loop above]    node {$0$}   ()
    (q0) edge                 node {$1$}   (q1)
    
    (q1) edge [bend left=15]  node {$0$}   (q2)
    (q1) edge [left]          node {$1$}   (q3)
    
    (q2) edge [bend left=15]  node {$0$}   (q1)
    (q2) edge [left]          node {$1$}   (q4)
    
    (q3) edge [loop below]    node {$0$}   ()
    (q3) edge                 node {$1$}   (q4)
    
    (q4) edge [loop below]    node {$0,1$} ()
    ;
    
  \end{tikzpicture}
\end{center}


%------------------------------------------------------------------------------
\newpage
\textsc{Primes}

Note that a number $n$ is said to be a \textbf{prime} if it is a whole number
that can only be divided by $1$ and itself
(i.e., $n$).
The positive integer $n$ is said to be \textbf{composite} if it is
greater than $1$ and is not a prime, which means
that it is possible to write $n$ as a product, $n = a \cdot b$,
where $a$ and $b$ are positive integers such that
$1 < a < n$ and $1 < b < n$.
A positive integer $n$ (positive means $> 0$) must fall into exactly
one of the 3 cases:
\begin{tightlist}
  \li $n$ is 1
  \li $n$ is prime
  \li $n$ is composite.
\end{tightlist}

Note that if $a$ and $b$ are integers and $a > 0$, $b > 0$,
then $a \leq ab$ and $b \leq ab$.
And if $a > 1$, then $b < ab$.



%------------------------------------------------------------------------------
\newpage
Q\textred{6}.
Consider the following fact:
Every positive integer $n \geq 1$ is a product of primes.
For instance for $n = 20$,
the ordered collection of primes involved are $(2, 2, 5)$ (in ascending order).
We define the product of the empty collection of primes, i.e. $()$, to be $1$.
Of course the collection can have one single number:
If the collection is $(11)$, then the product is 11.

Define $P(n)$ to be the above statement, i.e.,
\[
  P(n) =
  \biggl(
  \text{Every positive integer $n \geq 1$ is a product of primes}
  \biggr)
\]
Give a complete proof of $P(n)$ for all $n \geq 1$
using strong or weak induction.
Here, I'm giving you the $P(n)$ for free.

You can assume Euclid's lemma: If $p$ is a prime dividing $ab$,
then $p$ divides $a$ or $p$ divides $b$.

(In fact not only is every positive integer $n$ representable
as a product of primes,
it is represented as a product of primes in a \textit{unique way}.
In other words, the ordered collection of primes for $n$ is unique.
This can also be proven using induction.
But you don't have to prove it.)

\SOLUTION

(a)
Let $M = (\Sigma, Q, q_0, F, \delta)$.
Define $\overline{M}$ to be the DFA
\[
\overline{M} = (\Sigma, Q, q_0, Q - F, \delta)
\]
We will show that
\[
L(\overline{M}) = \overline{L(M)}
\]
Let $w \in \Sigma^*$.
\begin{align*}
  w \in L(\overline{M})
  &\iff ? \\
  &\iff ? \\
  &\iff w \in \overline{L(M)} \\
\end{align*}

(Recall that $w \in L(M)$ iff $\delta^*(q_0, w) \in F$.)

(b)


%------------------------------------------------------------------------------
\newpage
Q\textred{7}.
Using mathematical induction, 
prove that for $n \geq 4$, $2^n \geq n^2$.

(Actually this can also be proven using Calculus.)

\SOLUTION


(a)

(b)

(c)



%------------------------------------------------------------------------------
\newpage
Q\textred{8}. Prove or disprove the following: Let $X$ and $Y$ be sets.
\begin{enumerate}
  \item[(a)] $P(X \times Y) = P(X) \times P(Y)$
  \item[(b)] $P(X \cup Y) = P(X) \cup P(Y)$
  \item[(c)] $P(X) = 2^{|X|}$ if $X$ is a finite set.
\end{enumerate}
Remember that to disprove a statement, all you need is a
counterexample.
When constructing examples, always construct the simplest.

\SOLUTION

(a) $T$ derives strings containing any number of $0$s (including none) and
exactly one $\#$.
Therefore $TT$ derives strings with any number of $0$s and exactly two
$\#$s.
$U$ derives strings of the form $0^n \# 0^{2n}$.
Hence $L(G)$ contains strings with any number of $0$s and exactly two
$\#$ or strings of the form $0^n \# 0^{2n}$, i.e.,
\[
L(G) = L(0^* \# 0^* \# 0^*) \cup \{0^n \# 0^{2n} \mid n \geq 0\} 
\]


%------------------------------------------------------------------------------
\newpage
Q\textred{9}.
Using mathematical induction, prove that an undirected simple graph
with $n$ nodes can have at most $n(n + 1)/2$ edges.
(Simple means there are no loop edges and no multiedges.
A loop edge is an edge that joins the same node.
Two edges are multiedges is the join the same two nodes.)
$P(n)$ is the statement
\lq\lq if $G$ is an undirected simple graph with $n$ nodes, then the
$G$ has at most $n(n+1)/2$ edges".

\SOLUTION

\input{q09.tex}

%------------------------------------------------------------------------------
\newpage
Q\textred{10}.
A triangulation graph is a graph which looks like a patchwork of 
 triangles, i.e., 
 the graph is constructed by \lq\lq glueing'' triangles
 on a tabletop. 
\begin{center}
\begin{tikzpicture}[shorten >=1pt,node distance=2cm,auto,initial text=]
\node   (a) [shape=circle, draw] at (0, 0) {};
\node   (b) [shape=circle, draw] at (1, 2) {};
\node   (c) [shape=circle, draw] at (2, 0) {};
\path[-] (a) edge node {} (b)
(b) edge node {} (c)
(c) edge node {} (a)
;
\end{tikzpicture}
\end{center}
along edges.
 Note that line (edge) crossings are not allowed.
 For instance here's one 5 nodes (i.e. circles) and 7 edges (i.e. lines):

\begin{center}
\begin{tikzpicture}[shorten >=1pt,node distance=2cm,auto,initial text=]
\node   (a) [shape=circle, draw] at (0, 3) {};
\node   (b) [shape=circle, draw] at (2, 3) {};
\node   (c) [shape=circle, draw] at (4, 2.5) {};
\node   (d) [shape=circle, draw] at (1, 0) {};
\node   (e) [shape=circle, draw] at (3, 1) {};
\path[-] (a) edge node {} (b)
(b) edge node {} (c)
(a) edge node {} (d)
(b) edge node {} (d)
(b) edge node {} (e)
(d) edge node {} (e)
(c) edge node {} (e)
;
\end{tikzpicture}
\end{center}

(a) Is it true that using the above construction (i.e., glueing
triangle graphs along edges), all the \lq\lq pieces'' (i.e. finite regions) 
that you see 
are always triangles (i.e. a shape that is bounded by 
exactly 3 edges)? For instance in the above graph, there are 3 triangles.

(b) The degree of a node is just the number of edges (lines) joined to it.
For the above type of graphs is it possible to draw one 
with exactly one node of degree 10? 
Show that it is possible by drawing such a graph or prove it's not possible.

(c) Is it possible to have a graph with 5 nodes of degree 2? 
Show that it is possible by drawing such a graph or prove it's not possible.

(d) Is it possible to have a graph with no nodes of degree 2? 
Show that it is possible by drawing such a graph or prove it's not possible.

(e) The degrees of the nodes in the above example is 2, 2, 3, 3, 4.
Is it possible to have a graph with degrees 2, 2, 2, 2, 4?
Show that it is possible by drawing such a graph or prove it's not possible.

(f) The degrees of the nodes in the above example is 2, 2, 3, 3, 4.
Is it possible to have a graph with degrees 2, 2, 3, 3, 3?
Show that it is possible by drawing such a graph or prove it's not possible.

(g) Is it possible to have a triangulation with no nodes of degree 2 or 3?

(h) Suppose you have a triangular graph $G$ with a node $v$ of degree 2.
If there are at least 4 nodes in $G$, after removing $v$ from $G$
and removing all the edges attached to $v$, is the resulting graph
a triangulation graph? 
Prove that this is true or provide a counterexample.
       
\SOLUTION

\input{q10.tex}

%------------------------------------------------------------------------------
\newpage
Q\textred{11}.
Sipser 0.13.

\SOLUTION

\input{q11.tex}

\end{document}

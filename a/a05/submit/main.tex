\input{myquizpreamble}
\input{yliow}
\input{ciss362}
\textwidth=6in

\renewcommand\TITLE{Assignment a05}




\renewcommand\AUTHOR{John Doe}

\begin{document}
\topmatter

\textsc{Objectives}
\begin{myenum}
\li Design DFAs
\li Design NFAs.
\end{myenum}

As before modify the files
\verb!q01.tex! for Q1,
\verb!q02.tex! for Q2, etc.

\begin{myenum}
\li Sipser 1.8: Q1.
\li Sipser 1.9: Q2.
\li Sipser 1.10: Q3.
\li Sipser 1.11: DIY and then check the solution in the Sipser book.
\li Sipser 1.12: Skip.
\li Sipser 1.13: Q4.
\li Sipser 1.14: Q5.
\li Sipser 1.15: Q6.
\li Sipser 1.16: Q7.
\end{myenum}

%------------------------------------------------------------------------------
\newpage
\textsc{How to draw a state diagram}

Here's an example showing you how to draw the elements of a state diagram.
Also, look at the solution to 1.3 below.

\begin{center}
  \begin{tikzpicture}[shorten >=1pt,>=triangle 60,double distance=2pt,node distance=2cm,auto,initial text=]
    
    \node[state,initial]   (q0) at (0, 0) {$q_0$};
    \node[state]           (q1) at (4, 0) {$q_1$};
    \node[state,accepting] (q2) at (8, 0) {$q_2$};
    \node[state]           (q3) at (0,-4) {$q_3$};
    \node[state]           (q4) at (4,-4) {$q_4$};

    \node[state]           (q5) at (12, 0) {$q_5$};
    \node[state]           (q6) at (12,-4) {$q_6$};

    \path[->]
    (q0) edge                node {$1$} (q1)    
    (q1) edge [bend left=15] node {$0$} (q2)
    (q0) edge [left]  node {$\alpha$} (q3)
    (q1) edge [right] node {$\beta$}  (q4)
    (q1) edge [loop above]   node {$1$} (q1)
    (q2) edge [loop below]   node {$0$} ()
    (q2) edge [bend left=15] node {$1$} (q1)
    (q3) edge [loop right]   node {$0,1$} ()
    (q4) edge [loop left]    node {$2,3$} ()
    (q5) edge [right, bend left=30]    node {$\gamma$} (q6)
    (q5) edge [right, bend left=60]    node {$\ep$}    (q6)
    (q5) edge [left, bend right=30]    node {$\delta$} (q6)
    ;
  \end{tikzpicture}
\end{center}

For more information on drawing state diagrams go to my website
and look for \verb!latex-automata.pdf!.
Let me know if you have any questions about drawing state diagram.

%------------------------------------------------------------------------------
\newpage
Q1. Sipser 1.8.

\textsc{Solution}.

(a.) %{w| w begins with a 1 and ends with a 0} \cup {w| w contains at least three 1s}
\begin{center}
  \begin{tikzpicture}[shorten >=1pt,>=triangle 60,double distance=2pt,node distance=2cm,auto,initial text=]

    \node[state,initial]   (q0)  at (0, 0)  {$q_0$};
    
    \node[state]           (q00)  at (2, 2)  {$q_00$};
    \node[state]           (q01)  at (4, 2)  {$q_01$};
    \node[state,accepting] (q02)  at (6, 2)  {$q_02$};
    
    \node[state]           (q10)  at (2, -2)  {$q_10$};
    \node[state]           (q11)  at (4, -2)  {$q_11$};
    \node[state]           (q12)  at (6, -2)  {$q_12$};
    \node[state,accepting] (q13)  at (8, -2)  {$q_13$};

    \path[->]
    (q0)  edge [bend left=30]  node {$\epsilon $} (q00)
    (q0)  edge [bend right=30] node {$\epsilon $} (q10)

    %--------------------------------------------------------    
    (q00) edge              node {$1$}      (q01)
   
    (q01) edge [bend left=15] node {$0$}    (q02)
    (q01) edge [loop above]   node {$1$}    ()
   
    (q02) edge [loop above]   node {$0$}    ()
    (q02) edge [bend left=15] node {$1$}    (q01)

    %--------------------------------------------------------    
    (q10) edge [loop above]   node {$0$}    ()
    (q10) edge                node {$1$}    (q11)
   
    (q11) edge [loop above]   node {$0$}    ()
    (q11) edge                node {$1$}    (q12)
   
    (q12) edge [loop above]   node {$0$}    ()
    (q12) edge                node {$1$}    (q13)

    (q13) edge [loop above]   node {$0,1$} ()
    ;
    
  \end{tikzpicture}
\end{center}

(b.) %{w| w contains the substring 0101 (i.e., w = x0101y for some x and y)} \cup {w| w doesn’t contain the substring 110}
\begin{center}
  \begin{tikzpicture}[shorten >=1pt,>=triangle 60,double distance=2pt,node distance=2cm,auto,initial text=]

    \node[state,initial]   (q0)  at (0, 0)  {$q_0$};
    
    \node[state]           (q00)  at (0, 2)  {$q_00$};
    \node[state]           (q01)  at (2, 2)  {$q_01$};
    \node[state,accepting] (q02)  at (4, 2)  {$q_02$};
    \node[state,accepting] (q03)  at (6, 2)  {$q_03$};
    \node[state,accepting] (q04)  at (8, 2)  {$q_04$};
    
    \node[state,accepting] (q10)  at (0, -2)  {$q_10$};
    \node[state,accepting] (q11)  at (2, -2)  {$q_11$};
    \node[state,accepting] (q12)  at (4, -2)  {$q_12$};
    \node[state]           (q13)  at (6, -2)  {$q_13$};

    \path[->]
    (q0)  edge [left] node {$\epsilon $} (q00)
    (q0)  edge [left] node {$\epsilon $} (q10)

    %--------------------------------------------------------    
    (q00) edge                 node {$0$}   (q01)
    (q00) edge [loop above]    node {$1$}   ()
    
    (q01) edge                 node {$0$}   (q02)
    (q01) edge [loop above]    node {$0$}   ()
    
    (q02) edge                 node {$0$}   (q03)
    (q02) edge [bend left=45]  node {$1$}   (q00)
    
    (q03) edge [bend left=45]  node {$0$}   (q01)
    (q03) edge                 node {$1$}   (q04)
    
    (q04) edge [loop above]    node {$0,1$} ()
    %--------------------------------------------------------    
    (q10) edge [loop below]    node {$0$}   ()
    (q10) edge                 node {$1$}   (q11)
    
    (q11) edge [bend left=30]  node {$0$}   (q10)
    (q11) edge                 node {$1$}   (q12)
    
    (q12) edge                 node {$0$}   (q13)
    (q12) edge [loop above]    node {$1$}   ()

    (q13) edge [loop above]    node {$0,1$} ()
    ;

  \end{tikzpicture}
\end{center}

%------------------------------------------------------------------------------
\newpage
Q2. Sipser 1.9.

\textsc{Solution}.

(a.) %{w| the length of w is at most 5} \dot {w| every odd position of w is a 1}
\begin{center}
  \begin{tikzpicture}[shorten >=1pt,>=triangle 60,double distance=2pt,node distance=2cm,auto,initial text=]

    \node[state,initial]   (q00)  at (0, 0)  {$q_00$};
    \node[state]           (q01)  at (2, 0)  {$q_01$};
    \node[state]           (q02)  at (4, 0)  {$q_02$};
    \node[state]           (q03)  at (6, 0)  {$q_03$};
    \node[state]           (q04)  at (8, 0)  {$q_04$};
    \node[state]           (q05)  at (10, 0) {$q_05$};
    \node[state]           (q06)  at (12, 0) {$q_06$};
    

    \node[state,accepting] (q10)  at (0, -4) {$q_10$};
    \node[state,accepting] (q11)  at (2, -4) {$q_11$};
    \node[state,accepting] (q12)  at (4, -4) {$q_12$};
    \node[state]           (q13)  at (6, -4) {$q_13$};

    \path[->]
    (q00) edge                 node {$0,1$}      (q01)
    (q00) edge [left]          node {$\epsilon$} (q10)
    
    (q01) edge                 node {$0,1$}      (q02)
    (q01) edge [bend right=15] node {$\epsilon$} (q10)
   
    (q02) edge                 node {$0,1$}      (q03)
    (q02) edge [bend right=15] node {$\epsilon$} (q10)
   
    (q03) edge                 node {$0,1$}      (q04)
    (q03) edge [bend right=15] node {$\epsilon$} (q10)
   
    (q04) edge                 node {$0,1$}      (q05)
    (q04) edge [bend right=15] node {$\epsilon$} (q10)
   
    (q05) edge                 node {$0,1$}      (q06)
    (q05) edge [bend right=15] node {$\epsilon$} (q10)

    (q06) edge [loop above]    node {$0,1$}      ()

    %-----------------------------------------------------
    (q10) edge                 node {$1$}        (q11)
    
    (q11) edge [bend left=15]  node {$0,1$}      (q12)
    
    (q12) edge                 node {$0$}        (q13)
    (q12) edge [bend left=15]  node {$1$}        (q11)
    
    (q13) edge [loop above]    node {$0,1$}      (q11)
    ;
    
  \end{tikzpicture}
\end{center}


(b.) %{w| w contains at least three 1s} \dot {}
\begin{center}
  \begin{tikzpicture}[shorten >=1pt,>=triangle 60,double distance=2pt,node distance=2cm,auto,initial text=]
    
    \node[state,initial]   (q0)  at (0, 0)  {$q_0$};
    \node[state]           (q1)  at (2, 0)  {$q_1$};
    \node[state]           (q2)  at (4, 0)  {$q_2$};
    \node[state]           (q3)  at (6, 0)  {$q_3$};
    \node[state,accepting] (q4)  at (8, 0)  {$q_4$};
    
    \path[->]
    (q0) edge [loop above]    node {$0$}        ()
    (q0) edge                 node {$1$}        (q1)

    (q1) edge [loop above]    node {$0$}        ()
    (q1) edge                 node {$1$}        (q2)

    (q2) edge [loop above]    node {$0$}        ()
    (q2) edge                 node {$1$}        (q3)

    (q3) edge [loop above]    node {$0$}        ()
    (q3) edge                 node {$\epsilon$} (q4)
    ;

  \end{tikzpicture}
\end{center}

%------------------------------------------------------------------------------
\newpage
Q3. Sipser 1.10.

\textsc{Solution}.


(b)
\begin{center}
  \begin{tikzpicture}[shorten >=1pt,>=triangle 60,double distance=2pt,node distance=2cm,auto,initial text=]
    
    \node[state,initial]   (q0) at (0, 0) {$q_0$};
    \node[state]           (q1) at (2, 0) {$q_0$};
    \node[state]           (q2) at (4, 0) {$q_0$};
    \node[state]           (q3) at (6, 0) {$q_0$};
    \node[state,accepting] (q4) at (8, 0) {$q_0$};

    \path[->]
    (q0) edge                 node {$0$}   (q1)
    (q0) edge [loop above]    node {$1$}   ()
    
    (q1) edge [bend left=45]  node {$0$}   (q0)
    (q1) edge                 node {$1$}   (q2)
    
    (q2) edge                 node {$0$}   (q3)
    (q2) edge [bend right=45] node {$1$}   (q0)
    
    (q3) edge [bend left=45]  node {$0$}   (q0)
    (q3) edge                 node {$1$}   (q4)
    
    (q4) edge [loop right]    node {$1,0$} ()
    ;
  \end{tikzpicture}
\end{center}

(c)
\begin{center}
  \begin{tikzpicture}[shorten >=1pt,>=triangle 60,double distance=2pt,node distance=2cm,auto,initial text=]
    
    \node[state,initial,accepting] (q0) at (0, 0)  {$q_0$};
    \node[state]                   (q1) at (0, -4) {$q_1$};
    \node[state,accepting]         (q2) at (3, 0)  {$q_2$};
    \node[state]                   (q3) at (3, -4) {$q_3$};
    \node[state,accepting]         (q4) at (6, 0)  {$q_4$};
    \node[state,accepting]         (q5) at (6, -4) {$q_5$};
    
    \path[->]
    (q0) edge [bend left=15]          node {$0$}   (q1)
    (q0) edge                         node {$1$}   (q2)

    (q1) edge [bend left=15]          node {$0$}   (q0)
    (q1) edge                         node {$1$}   (q3)

    (q2) edge [bend left=15]          node {$0$}   (q3)
    (q2) edge                         node {$1$}   (q4)

    (q3) edge [bend left=15]          node {$0$}   (q2)
    (q3) edge                         node {$1$}   (q5)
    
    (q4) edge [bend left=15]          node {$0$}   (q5)
    
    (q5) edge [bend left=15]          node {$0$}   (q4)
    ;
  \end{tikzpicture}
\end{center}

(d)
\begin{center}
  \begin{tikzpicture}[shorten >=1pt,>=triangle 60,double distance=2pt,node distance=2cm,auto,initial text=]
    
    \node[state,initial]   (q0) at (0, 0) {$q_0$};
    \node[state,accepting] (q1) at (4, 0) {$q_1$};

    \path[->]
    (q0) edge node {$0$}   (q1)
    ;
  \end{tikzpicture}
\end{center}



(e)
\begin{center}
  \begin{tikzpicture}[shorten >=1pt,>=triangle 60,double distance=2pt,node distance=2cm,auto,initial text=]
    
    \node[state,initial]   (q0) at (0, 0) {$q_0$};
    \node[state,accepting] (q1) at (3, 0) {$q_1$};
    \node[state]           (q2) at (6, 0) {$q_2$};

    \path[->]
    (q0) edge                 node {$0$} (q1)
    (q0) edge [bend right=30] node {$1$} (q2)
    
    (q1) edge [loop above]    node {$0$} ()
    (q1) edge [bend left=15]  node {$1$} (q2)
    
    (q2) edge [bend left=15]  node {$0$} (q1)
    (q2) edge [loop right]    node {$1$} ()

    ;
  \end{tikzpicture}
\end{center}

\newpage
(g)
\begin{center}
  \begin{tikzpicture}[shorten >=1pt,>=triangle 60,double distance=2pt,node distance=2cm,auto,initial text=]
    
    \node[state,initial,accepting] (q0) at (0, 0) {$q_0$};

    \path[->]
    ;
  \end{tikzpicture}
\end{center}

(h)
\begin{center}
  \begin{tikzpicture}[shorten >=1pt,>=triangle 60,double distance=2pt,node distance=2cm,auto,initial text=]
    
    \node[state,initial,accepting] (q0) at (0, 0) {$q_0$};

    \path[->]
    (q0) edge [loop right] node {$0$} ()
    ;
  \end{tikzpicture}
\end{center}


%------------------------------------------------------------------------------
\newpage
Q4. Sipser 1.12.

\textsc{Solution}.

Regular Expression: $D = \{\beta(\beta\beta)^*(\alpha\alpha)^*\}$

\begin{center}
  \begin{tikzpicture}[shorten >=1pt,>=triangle 60,double distance=2pt,node distance=2cm,auto,initial text=]
    
    \node[state,initial]   (q0) at (0,0) {$q_0$};
    \node[state,accepting] (q1) at (2,0) {$q_1$};
    \node[state]           (q2) at (4,0) {$q_2$};
    \node[state,accepting] (q3) at (6,0) {$q_3$};
    \node[state]           (q4) at (8,0) {$q_4$};
    
    \path[->]
    (q0) edge [bend right=45] node {$\alpha$}       (q4)
    (q0) edge [bend left=15]  node {$\beta$}        (q1)
    
    (q1) edge                 node {$\alpha$}       (q2)
    (q1) edge [bend left=15]  node {$\beta$}        (q0)
    
    (q2) edge [bend left=15]  node {$\alpha$}       (q3)
    (q2) edge [bend right=45] node {$\beta$}        (q4)
    
    (q3) edge [bend left=15]  node {$\alpha$}       (q2)
    (q3) edge                 node {$\beta$}        (q4)
    
    (q4) edge [loop right]    node {$\alpha,\beta$} ()
    ;
    
  \end{tikzpicture}
\end{center}

%------------------------------------------------------------------------------
Q5. Sipser 1.13.

\textsc{Solution}.

\begin{center}
  \begin{tikzpicture}[shorten >=1pt,>=triangle 60,double distance=2pt,node distance=2cm,auto,initial text=]

    \node[state,initial,accepting] (q0)  at (0, 0)  {$q_0$};
    \node[state,accepting]         (q1)  at (2, 0)  {$q_1$};
    \node[state,accepting]         (q2)  at (4, 0)  {$q_2$};
    \node[state,accepting]         (q3)  at (2, -2) {$q_3$};
    \node[state]                   (q4)  at (4, -2) {$q_4$};

    \path[->]
    (q0) edge [loop above]    node {$0$}   ()
    (q0) edge                 node {$1$}   (q1)
    
    (q1) edge [bend left=15]  node {$0$}   (q2)
    (q1) edge [left]          node {$1$}   (q3)
    
    (q2) edge [bend left=15]  node {$0$}   (q1)
    (q2) edge [left]          node {$1$}   (q4)
    
    (q3) edge [loop below]    node {$0$}   ()
    (q3) edge                 node {$1$}   (q4)
    
    (q4) edge [loop below]    node {$0,1$} ()
    ;
    
  \end{tikzpicture}
\end{center}


%------------------------------------------------------------------------------

\newpage
Q6. Sipser 1.14.

For (b) make sure you example is the simplest.

\textsc{Solution}.

(a)
Let $M = (\Sigma, Q, q_0, F, \delta)$.
Define $\overline{M}$ to be the DFA
\[
\overline{M} = (\Sigma, Q, q_0, Q - F, \delta)
\]
We will show that
\[
L(\overline{M}) = \overline{L(M)}
\]
Let $w \in \Sigma^*$.
\begin{align*}
  w \in L(\overline{M})
  &\iff ? \\
  &\iff ? \\
  &\iff w \in \overline{L(M)} \\
\end{align*}

(Recall that $w \in L(M)$ iff $\delta^*(q_0, w) \in F$.)

(b)


\end{document}

\input{myquizpreamble}
\input{yliow}
\input{ciss362}
\textwidth=6in

\renewcommand\TITLE{Assignment a05}




\renewcommand\AUTHOR{John Doe}

\begin{document}
\topmatter


In general, if a solution is provided in Sipser, you should try it first
and then check against they solution.
If I provide a solution, do the same before looking at my solution.

An example of how to draw state diagrams using \LaTeX\ is given below.
Also, study the solution and the \LaTeX\ code to 1.3 below.

\begin{enumerate}
\li Sipser 1.1: The solution is in the Sipser book.
\li Sipser 1.2: The solution is in the Sipser book.
\li Sipser 1.3: The solution is provided below.

\li Sipser 1.4: Q1. You solve Sipser 1.4 except for (b) and (d).
Enter your answer in \verb!q01.tex!.
Solution to 1.4(b), 1.4(d) can be found in Sipser. 

\li Sipser 1.5: Q2. You solve Sipser 1.5 except for (a) and (b).
Enter your answer in \verb!q02.tex!.
Solution to 1.5(a), 1.5(b) can be found in Sipser.
\begin{enumerate}[nosep]
\item In 1.5(d), $a^{*}b^{*}$ is a shorthand for $\{a\}^{*} \{b\}^{*}$.
\item In 1.5(e), $(ab^{+})^{*}$ is a shorthand for $(\{a\}\{b\}^{+})^{*}$
where $\{b\}^+ = \{b^n \mid n \geq 1\}$
\item In 1.5(f), $a^{*} \cup b^{*}$ is a shorthand for $\{a\}^{*} \cup \{b\}^{*}$.
\end{enumerate}

\li Sipser 1.6: DIY.

\li Sipser 1.7: Q3.
You solve Sipser 1.7 except for (a) and (f).
Enter your answer in \verb!q03.tex!.
Solution to 1.7(a), 1.7(f) can be found in Sipser.
\end{enumerate}


\newpage
\textsc{How to draw a state diagram}

Here's an example showing you how to draw the elements of a state diagram.
Also, look at the solution to 1.3 below.

\begin{center}
  \begin{tikzpicture}[shorten >=1pt,>=triangle 60,double distance=2pt,node distance=2cm,auto,initial text=]
    \node[state,initial]   (q0) at (0, 0) {$q_0$};
    \node[state]           (q1) at (4, 0) {$q_1$};
    \node[state,accepting] (q2) at (8, 0) {$q_2$};
    \node[state]           (q3) at (0,-4) {$q_3$};
    \node[state]           (q4) at (4,-4) {$q_4$};

    \node[state]           (q5) at (12, 0) {$q_5$};
    \node[state]           (q6) at (12,-4) {$q_6$};

    \path[->]
    (q0) edge                node {$1$} (q1)    
    (q1) edge [bend left=15] node {$0$} (q2)
    (q0) edge [left]  node {$\alpha$} (q3)
    (q1) edge [right] node {$\beta$}  (q4)
    (q1) edge [loop above]   node {$1$} (q1)
    (q2) edge [loop below]   node {$0$} ()
    (q2) edge [bend left=15] node {$1$} (q1)
    (q3) edge [loop right]   node {$0,1$} ()
    (q4) edge [loop left]    node {$2,3$} ()
    (q5) edge [right, bend left=30]    node {$\gamma$} (q6)
    (q5) edge [right, bend left=60]    node {$\ep$}    (q6)
    (q5) edge [left, bend right=30]    node {$\delta$} (q6)
    ;
  \end{tikzpicture}
\end{center}

For more information on drawing state diagrams go to my website,
scroll down to the Tutorials section and
look for \verb!latex-automata.pdf!.

Let me know if you have any questions about drawing state diagram.

\newpage

Solution to Sipser 1.3.


\textsc{Solution}.

\begin{center}
\begin{tikzpicture}[shorten >=1pt,node distance=2cm,auto,initial text=,
                    >=triangle 60, initial where=above]
\node[state]                   (q_1)          {$q_1$};
\node[state]                   (q_2) at (2,0) {$q_2$};
\node[state,initial,accepting] (q_3) at (4,0) {$q_3$};
\node[state]                   (q_4) at (6,0) {$q_4$};
\node[state]                   (q_5) at (8,0) {$q_5$};


\path[->] (q_1) edge [loop below] node {$u$} ()
          (q_1) edge [bend left]  node {$d$} (q_2)

          (q_2) edge [bend left]  node {$u$} (q_1)
          (q_2) edge [bend left]  node {$d$} (q_3)

          (q_3) edge [bend left]  node {$u$} (q_2)
          (q_3) edge [bend left]  node {$d$} (q_4)

          (q_4) edge [bend left]  node {$u$} (q_3)
          (q_4) edge [bend left]  node {$d$} (q_5)

          (q_5) edge [bend left]  node {$u$} (q_4)
          (q_5) edge [loop below]  node {$d$} ()
;
\end{tikzpicture}
\end{center}



%-------------------------------------------------------------------------------
\newpage
Q1. Sipser 1.4 except for (b) and (d).

\textsc{Solution}.

(a.) %{w| w begins with a 1 and ends with a 0} \cup {w| w contains at least three 1s}
\begin{center}
  \begin{tikzpicture}[shorten >=1pt,>=triangle 60,double distance=2pt,node distance=2cm,auto,initial text=]

    \node[state,initial]   (q0)  at (0, 0)  {$q_0$};
    
    \node[state]           (q00)  at (2, 2)  {$q_00$};
    \node[state]           (q01)  at (4, 2)  {$q_01$};
    \node[state,accepting] (q02)  at (6, 2)  {$q_02$};
    
    \node[state]           (q10)  at (2, -2)  {$q_10$};
    \node[state]           (q11)  at (4, -2)  {$q_11$};
    \node[state]           (q12)  at (6, -2)  {$q_12$};
    \node[state,accepting] (q13)  at (8, -2)  {$q_13$};

    \path[->]
    (q0)  edge [bend left=30]  node {$\epsilon $} (q00)
    (q0)  edge [bend right=30] node {$\epsilon $} (q10)

    %--------------------------------------------------------    
    (q00) edge              node {$1$}      (q01)
   
    (q01) edge [bend left=15] node {$0$}    (q02)
    (q01) edge [loop above]   node {$1$}    ()
   
    (q02) edge [loop above]   node {$0$}    ()
    (q02) edge [bend left=15] node {$1$}    (q01)

    %--------------------------------------------------------    
    (q10) edge [loop above]   node {$0$}    ()
    (q10) edge                node {$1$}    (q11)
   
    (q11) edge [loop above]   node {$0$}    ()
    (q11) edge                node {$1$}    (q12)
   
    (q12) edge [loop above]   node {$0$}    ()
    (q12) edge                node {$1$}    (q13)

    (q13) edge [loop above]   node {$0,1$} ()
    ;
    
  \end{tikzpicture}
\end{center}

(b.) %{w| w contains the substring 0101 (i.e., w = x0101y for some x and y)} \cup {w| w doesn’t contain the substring 110}
\begin{center}
  \begin{tikzpicture}[shorten >=1pt,>=triangle 60,double distance=2pt,node distance=2cm,auto,initial text=]

    \node[state,initial]   (q0)  at (0, 0)  {$q_0$};
    
    \node[state]           (q00)  at (0, 2)  {$q_00$};
    \node[state]           (q01)  at (2, 2)  {$q_01$};
    \node[state,accepting] (q02)  at (4, 2)  {$q_02$};
    \node[state,accepting] (q03)  at (6, 2)  {$q_03$};
    \node[state,accepting] (q04)  at (8, 2)  {$q_04$};
    
    \node[state,accepting] (q10)  at (0, -2)  {$q_10$};
    \node[state,accepting] (q11)  at (2, -2)  {$q_11$};
    \node[state,accepting] (q12)  at (4, -2)  {$q_12$};
    \node[state]           (q13)  at (6, -2)  {$q_13$};

    \path[->]
    (q0)  edge [left] node {$\epsilon $} (q00)
    (q0)  edge [left] node {$\epsilon $} (q10)

    %--------------------------------------------------------    
    (q00) edge                 node {$0$}   (q01)
    (q00) edge [loop above]    node {$1$}   ()
    
    (q01) edge                 node {$0$}   (q02)
    (q01) edge [loop above]    node {$0$}   ()
    
    (q02) edge                 node {$0$}   (q03)
    (q02) edge [bend left=45]  node {$1$}   (q00)
    
    (q03) edge [bend left=45]  node {$0$}   (q01)
    (q03) edge                 node {$1$}   (q04)
    
    (q04) edge [loop above]    node {$0,1$} ()
    %--------------------------------------------------------    
    (q10) edge [loop below]    node {$0$}   ()
    (q10) edge                 node {$1$}   (q11)
    
    (q11) edge [bend left=30]  node {$0$}   (q10)
    (q11) edge                 node {$1$}   (q12)
    
    (q12) edge                 node {$0$}   (q13)
    (q12) edge [loop above]    node {$1$}   ()

    (q13) edge [loop above]    node {$0,1$} ()
    ;

  \end{tikzpicture}
\end{center}



%-------------------------------------------------------------------------------
\newpage
Q2. Sipser 1.5 except for (a) and (b).

\textsc{Solution}.

(a.) %{w| the length of w is at most 5} \dot {w| every odd position of w is a 1}
\begin{center}
  \begin{tikzpicture}[shorten >=1pt,>=triangle 60,double distance=2pt,node distance=2cm,auto,initial text=]

    \node[state,initial]   (q00)  at (0, 0)  {$q_00$};
    \node[state]           (q01)  at (2, 0)  {$q_01$};
    \node[state]           (q02)  at (4, 0)  {$q_02$};
    \node[state]           (q03)  at (6, 0)  {$q_03$};
    \node[state]           (q04)  at (8, 0)  {$q_04$};
    \node[state]           (q05)  at (10, 0) {$q_05$};
    \node[state]           (q06)  at (12, 0) {$q_06$};
    

    \node[state,accepting] (q10)  at (0, -4) {$q_10$};
    \node[state,accepting] (q11)  at (2, -4) {$q_11$};
    \node[state,accepting] (q12)  at (4, -4) {$q_12$};
    \node[state]           (q13)  at (6, -4) {$q_13$};

    \path[->]
    (q00) edge                 node {$0,1$}      (q01)
    (q00) edge [left]          node {$\epsilon$} (q10)
    
    (q01) edge                 node {$0,1$}      (q02)
    (q01) edge [bend right=15] node {$\epsilon$} (q10)
   
    (q02) edge                 node {$0,1$}      (q03)
    (q02) edge [bend right=15] node {$\epsilon$} (q10)
   
    (q03) edge                 node {$0,1$}      (q04)
    (q03) edge [bend right=15] node {$\epsilon$} (q10)
   
    (q04) edge                 node {$0,1$}      (q05)
    (q04) edge [bend right=15] node {$\epsilon$} (q10)
   
    (q05) edge                 node {$0,1$}      (q06)
    (q05) edge [bend right=15] node {$\epsilon$} (q10)

    (q06) edge [loop above]    node {$0,1$}      ()

    %-----------------------------------------------------
    (q10) edge                 node {$1$}        (q11)
    
    (q11) edge [bend left=15]  node {$0,1$}      (q12)
    
    (q12) edge                 node {$0$}        (q13)
    (q12) edge [bend left=15]  node {$1$}        (q11)
    
    (q13) edge [loop above]    node {$0,1$}      (q11)
    ;
    
  \end{tikzpicture}
\end{center}


(b.) %{w| w contains at least three 1s} \dot {}
\begin{center}
  \begin{tikzpicture}[shorten >=1pt,>=triangle 60,double distance=2pt,node distance=2cm,auto,initial text=]
    
    \node[state,initial]   (q0)  at (0, 0)  {$q_0$};
    \node[state]           (q1)  at (2, 0)  {$q_1$};
    \node[state]           (q2)  at (4, 0)  {$q_2$};
    \node[state]           (q3)  at (6, 0)  {$q_3$};
    \node[state,accepting] (q4)  at (8, 0)  {$q_4$};
    
    \path[->]
    (q0) edge [loop above]    node {$0$}        ()
    (q0) edge                 node {$1$}        (q1)

    (q1) edge [loop above]    node {$0$}        ()
    (q1) edge                 node {$1$}        (q2)

    (q2) edge [loop above]    node {$0$}        ()
    (q2) edge                 node {$1$}        (q3)

    (q3) edge [loop above]    node {$0$}        ()
    (q3) edge                 node {$\epsilon$} (q4)
    ;

  \end{tikzpicture}
\end{center}


%-------------------------------------------------------------------------------
\newpage
Q3. Sipser 1.7 except for (a) and (f).

\textsc{Solution}.


(b)
\begin{center}
  \begin{tikzpicture}[shorten >=1pt,>=triangle 60,double distance=2pt,node distance=2cm,auto,initial text=]
    
    \node[state,initial]   (q0) at (0, 0) {$q_0$};
    \node[state]           (q1) at (2, 0) {$q_0$};
    \node[state]           (q2) at (4, 0) {$q_0$};
    \node[state]           (q3) at (6, 0) {$q_0$};
    \node[state,accepting] (q4) at (8, 0) {$q_0$};

    \path[->]
    (q0) edge                 node {$0$}   (q1)
    (q0) edge [loop above]    node {$1$}   ()
    
    (q1) edge [bend left=45]  node {$0$}   (q0)
    (q1) edge                 node {$1$}   (q2)
    
    (q2) edge                 node {$0$}   (q3)
    (q2) edge [bend right=45] node {$1$}   (q0)
    
    (q3) edge [bend left=45]  node {$0$}   (q0)
    (q3) edge                 node {$1$}   (q4)
    
    (q4) edge [loop right]    node {$1,0$} ()
    ;
  \end{tikzpicture}
\end{center}

(c)
\begin{center}
  \begin{tikzpicture}[shorten >=1pt,>=triangle 60,double distance=2pt,node distance=2cm,auto,initial text=]
    
    \node[state,initial,accepting] (q0) at (0, 0)  {$q_0$};
    \node[state]                   (q1) at (0, -4) {$q_1$};
    \node[state,accepting]         (q2) at (3, 0)  {$q_2$};
    \node[state]                   (q3) at (3, -4) {$q_3$};
    \node[state,accepting]         (q4) at (6, 0)  {$q_4$};
    \node[state,accepting]         (q5) at (6, -4) {$q_5$};
    
    \path[->]
    (q0) edge [bend left=15]          node {$0$}   (q1)
    (q0) edge                         node {$1$}   (q2)

    (q1) edge [bend left=15]          node {$0$}   (q0)
    (q1) edge                         node {$1$}   (q3)

    (q2) edge [bend left=15]          node {$0$}   (q3)
    (q2) edge                         node {$1$}   (q4)

    (q3) edge [bend left=15]          node {$0$}   (q2)
    (q3) edge                         node {$1$}   (q5)
    
    (q4) edge [bend left=15]          node {$0$}   (q5)
    
    (q5) edge [bend left=15]          node {$0$}   (q4)
    ;
  \end{tikzpicture}
\end{center}

(d)
\begin{center}
  \begin{tikzpicture}[shorten >=1pt,>=triangle 60,double distance=2pt,node distance=2cm,auto,initial text=]
    
    \node[state,initial]   (q0) at (0, 0) {$q_0$};
    \node[state,accepting] (q1) at (4, 0) {$q_1$};

    \path[->]
    (q0) edge node {$0$}   (q1)
    ;
  \end{tikzpicture}
\end{center}



(e)
\begin{center}
  \begin{tikzpicture}[shorten >=1pt,>=triangle 60,double distance=2pt,node distance=2cm,auto,initial text=]
    
    \node[state,initial]   (q0) at (0, 0) {$q_0$};
    \node[state,accepting] (q1) at (3, 0) {$q_1$};
    \node[state]           (q2) at (6, 0) {$q_2$};

    \path[->]
    (q0) edge                 node {$0$} (q1)
    (q0) edge [bend right=30] node {$1$} (q2)
    
    (q1) edge [loop above]    node {$0$} ()
    (q1) edge [bend left=15]  node {$1$} (q2)
    
    (q2) edge [bend left=15]  node {$0$} (q1)
    (q2) edge [loop right]    node {$1$} ()

    ;
  \end{tikzpicture}
\end{center}

\newpage
(g)
\begin{center}
  \begin{tikzpicture}[shorten >=1pt,>=triangle 60,double distance=2pt,node distance=2cm,auto,initial text=]
    
    \node[state,initial,accepting] (q0) at (0, 0) {$q_0$};

    \path[->]
    ;
  \end{tikzpicture}
\end{center}

(h)
\begin{center}
  \begin{tikzpicture}[shorten >=1pt,>=triangle 60,double distance=2pt,node distance=2cm,auto,initial text=]
    
    \node[state,initial,accepting] (q0) at (0, 0) {$q_0$};

    \path[->]
    (q0) edge [loop right] node {$0$} ()
    ;
  \end{tikzpicture}
\end{center}


\end{document}
